\documentclass[man]{apa6}
\usepackage{lmodern}
\usepackage{amssymb,amsmath}
\usepackage{ifxetex,ifluatex}
\usepackage{fixltx2e} % provides \textsubscript
\ifnum 0\ifxetex 1\fi\ifluatex 1\fi=0 % if pdftex
  \usepackage[T1]{fontenc}
  \usepackage[utf8]{inputenc}
\else % if luatex or xelatex
  \ifxetex
    \usepackage{mathspec}
  \else
    \usepackage{fontspec}
  \fi
  \defaultfontfeatures{Ligatures=TeX,Scale=MatchLowercase}
\fi
% use upquote if available, for straight quotes in verbatim environments
\IfFileExists{upquote.sty}{\usepackage{upquote}}{}
% use microtype if available
\IfFileExists{microtype.sty}{%
\usepackage{microtype}
\UseMicrotypeSet[protrusion]{basicmath} % disable protrusion for tt fonts
}{}
\usepackage{hyperref}
\hypersetup{unicode=true,
            pdftitle={Regression models for Cylindrical data in Psychology},
            pdfauthor={Jolien Cremers, Helena J.M. Pennings, \& Christophe Ley},
            pdfkeywords={cylindrical data, regression, interpersonal behavior},
            pdfborder={0 0 0},
            breaklinks=true}
\urlstyle{same}  % don't use monospace font for urls
\usepackage{graphicx,grffile}
\makeatletter
\def\maxwidth{\ifdim\Gin@nat@width>\linewidth\linewidth\else\Gin@nat@width\fi}
\def\maxheight{\ifdim\Gin@nat@height>\textheight\textheight\else\Gin@nat@height\fi}
\makeatother
% Scale images if necessary, so that they will not overflow the page
% margins by default, and it is still possible to overwrite the defaults
% using explicit options in \includegraphics[width, height, ...]{}
\setkeys{Gin}{width=\maxwidth,height=\maxheight,keepaspectratio}
\IfFileExists{parskip.sty}{%
\usepackage{parskip}
}{% else
\setlength{\parindent}{0pt}
\setlength{\parskip}{6pt plus 2pt minus 1pt}
}
\setlength{\emergencystretch}{3em}  % prevent overfull lines
\providecommand{\tightlist}{%
  \setlength{\itemsep}{0pt}\setlength{\parskip}{0pt}}
\setcounter{secnumdepth}{0}
% Redefines (sub)paragraphs to behave more like sections
\ifx\paragraph\undefined\else
\let\oldparagraph\paragraph
\renewcommand{\paragraph}[1]{\oldparagraph{#1}\mbox{}}
\fi
\ifx\subparagraph\undefined\else
\let\oldsubparagraph\subparagraph
\renewcommand{\subparagraph}[1]{\oldsubparagraph{#1}\mbox{}}
\fi

%%% Use protect on footnotes to avoid problems with footnotes in titles
\let\rmarkdownfootnote\footnote%
\def\footnote{\protect\rmarkdownfootnote}


  \title{Regression models for Cylindrical data in Psychology}
    \author{Jolien Cremers\textsuperscript{1}, Helena J.M.
Pennings\textsuperscript{2,3}, \& Christophe Ley\textsuperscript{4}}
    \date{}
  
\shorttitle{Cylindrical data in Psychology}
\affiliation{
\vspace{0.5cm}
\textsuperscript{1} Department of Methodology and Statistics, Utrecht University\\\textsuperscript{2} TNO\\\textsuperscript{3} Department of Education, Utrecht University\\\textsuperscript{4} Department of Applied Mathematics, Computer Science and Statistics, Ghent University}
\keywords{cylindrical data, regression, interpersonal behavior}
\usepackage{csquotes}
\usepackage{upgreek}
\captionsetup{font=singlespacing,justification=justified}

\usepackage{longtable}
\usepackage{lscape}
\usepackage{multirow}
\usepackage{tabularx}
\usepackage[flushleft]{threeparttable}
\usepackage{threeparttablex}

\newenvironment{lltable}{\begin{landscape}\begin{center}\begin{ThreePartTable}}{\end{ThreePartTable}\end{center}\end{landscape}}

\makeatletter
\newcommand\LastLTentrywidth{1em}
\newlength\longtablewidth
\setlength{\longtablewidth}{1in}
\newcommand{\getlongtablewidth}{\begingroup \ifcsname LT@\roman{LT@tables}\endcsname \global\longtablewidth=0pt \renewcommand{\LT@entry}[2]{\global\advance\longtablewidth by ##2\relax\gdef\LastLTentrywidth{##2}}\@nameuse{LT@\roman{LT@tables}} \fi \endgroup}


\usepackage{lineno}

\linenumbers
\DeclareRobustCommand{\VANDER}[3]{#2}
\usepackage{multirow}
\usepackage{rotating}
\usepackage{color}
\usepackage{subcaption}

\authornote{

Correspondence concerning this article should be addressed to Jolien
Cremers, . E-mail:
\href{mailto:joliencremers@gmail.com}{\nolinkurl{joliencremers@gmail.com}}}

\abstract{
Cylindrical data are multivariate data which consist of a directional,
in this paper circular, and a linear component. Examples of cylindrical
data in psychology include human navigation (direction and distance of
movement), eye-tracking research (direction and length of saccades) and
data from an interpersonal circumplex (type and strength of
interpersonal behavior). In this paper we adapt four models for
cylindrical data to include a regression of the circular and linear
component onto a set of covariates. Subsequently, we illustrate how to
fit these models and interpret their results on a dataset on the
interpersonal behavior of teachers.


}

\usepackage{amsthm}
\newtheorem{theorem}{Theorem}[section]
\newtheorem{lemma}{Lemma}[section]
\theoremstyle{definition}
\newtheorem{definition}{Definition}[section]
\newtheorem{corollary}{Corollary}[section]
\newtheorem{proposition}{Proposition}[section]
\theoremstyle{definition}
\newtheorem{example}{Example}[section]
\theoremstyle{definition}
\newtheorem{exercise}{Exercise}[section]
\theoremstyle{remark}
\newtheorem*{remark}{Remark}
\newtheorem*{solution}{Solution}
\begin{document}
\maketitle

\section{Introduction}\label{Introduction}

In social sciences the use of cylindrical data is very common. Such data
consist of a linear and a circular component. Gurtman (2011) refers to
such data as vectors, with a directional measure (i.e., the circular
component) and a measure indicating the magnitude (i.e., linear
component). Many established models in psychology are often referred to
as circular or circumplex models, but those models are cylindrical.
Examples of such cylindrical models are the interpersonal
circle/circumplex (Leary, 1957; Wiggins, 1996; Wubbels, Brekelmans,
Brok, \& Tartwijk, 2006), the circumplex of affect (Russell, 1980), the
circumplex of human emotion (Plutchik, 1997) or the model of human
values (Schwartz, 1992). \newline \indent Also, many of the more recent
types of data that are studied in psychology are cylindrical. For
example, research on human navigation uses data where distance (i.e.,
linear component) and direction (i.e., the linear component) are of
interest (Chrastil \& Warren, 2017) or in eye-tracking, the saccade data
also consist of both the direction (i.e., the circular variable) and the
duration (i.e., the linear variable)(e.g., Rayner (2009)). Apart from
the social sciences data with a circular and linear outcome more
commonly occur in meteorology (García-Portugués, Crujeiras, \&
González-Manteiga, 2013), ecology (García-Portugués, Barros, Crujeiras,
González-Manteiga, \& Pereira, 2014) or marine research (Lagona, Picone,
Maruotti, \& Cosoli, 2015) \newline \indent Up until now researchers
studying cylindrical data had to rely on linear statistical methods to
analyze their research results. However, lately more and more of these
researchers acknowledge that linear methods are not sufficient and call
for new methods (Gurtman, 2011; Pennings, 2017b; Wright, Pincus, Conroy,
\& Hilsenroth, 2009) that take into account both the circular and the
linear component of these data. \newline \indent The aim in the present
paper is twofold. Firstly, we we intend to fill the above mentioned gap
in the literature by showing the use of cylindrical models can benefit
the analysis of circumplex data and cylindrical data in psychology in
general. More specifically we will show these benefits for interpersonal
teacher data from the field of educational psychology. Apart from
modelling the dependence between the linear and circular component of a
cylindrical variable we would also like to predict the two components
from a set of covariates in a regression model. Our second aim therefore
is to adapt several existing cylindrical models in such a way that they
include a regression of both the linear and circular component of a
cylindrical variable onto a set of covariates. From now on we will
therefore refer to the components of the cylindrical variable as outcome
components. These adapted circular models are then used to analyse the
teacher data.

\subsection{Cylindrical data}

Data that consist of a variable and a circular variable is called
cylindrical data. A circular variable is different from the linear
variable in the sense that it is measured on a different scale. Figure
\ref{circline} shows the difference between a circular scale (right) and
a linear scale (left). The most important difference is that on a
circular scale the datapoints 0\(^\circ\) and 360\(^\circ\) are
connected and in fact represent the same number while on a linear scale
the two ends, \(-\infty\) and \(\infty\), are not connected and
consequently the values 0\(^\circ\) and 360\(^\circ\) are located on
different places on the scale. Both circular data and cylindrical data
require special analysis methods due to this periodicity in the scale of
a circular variable (see e.g.~Fisher (1995) for an introduction to
circular data and Mardia and Jupp (2000), Jammalamadaka and Sengupta
(2001) and Ley and Verdebout (2017) for a more elaborate
overview).\newline \indent In the literature, several methods have been
put forward to model the relation between the linear and circular
component of a cylindrical variable. Some of these are based on
regressing the linear component onto the circular component using the
following type of relation: \[y = \beta_0 +
\beta_1*\cos(\theta) + \beta_2*\sin(\theta)+ \epsilon,\] where \(y\) is
the linear component and \(\theta\) the circular component (Johnson \&
Wehrly, 1978; Mardia \& Sutton, 1978; Mastrantonio, Maruotti, \&
Jona-Lasinio, 2015). Others model the relation in a different way,
e.g.~by specifying a multivariate model for several linear and circular
variables and modelling their covariance matrix (Mastrantonio, 2018) or
by proposing a joint cylindrical distribution. For example, Abe and Ley
(2017) introduce a cylindrical distribution based on a Weibull
distribution for the linear component and a sine-skewed von Mises
distribution for the circular component and link these through their
respective shape and concentration parameters. However, none of the
methods that have been proposed thus far include additional covariates
onto which both the circular and linear component are regressed.

\begin{figure}
\centering
\includegraphics[width = 0.8\textwidth]{Plots/circline.pdf}
\caption{The difference between a linear scale (left) and a circular scale (right).}
\label{circline}
\end{figure}

\section{Teacher Data}\label{Example}

\begin{figure}
\centering
\includegraphics[width = 0.8\textwidth]{Plots/IPC-T.png}
\caption{The interpersonal circle for teachers (IPC-T). The words presented in
the circumference of the circle are anchor words to describe the type of
behavior located in each part of the IPC.}
\label{QTI}
\end{figure}

The motivating example for this article comes from the field of
educational psychology and was collected for the studies on classroom
climate of \VANDER{Want}{Van der}{van der} Want (2015), Claessens (2016)
and Pennings (2017a). An indicator of the quality of the classroom
climate is the students' perception of their teachers' interpersonal
behavior. These interpersonal perceptions, both in educational
psychology as well as in other areas of psychology, can be measured
using circumplex measurement instruments (see Horowitz and Strack (2011)
for an overview of many such instruments).\newline \indent The
circumplex data used in this paper are measured using the Questionnaire
on Teacher Interaction (QTI) (Wubbels et al., 2006) which is one such
circumplex measurement instrument. The QTI is designed to measure
student perceptions of their teachers' interpersonal behavior and
contains items that load on two interpersonal dimensions: Agency and
Communion. Agency refers to the degree of power or control a teacher
exerts in interaction with his/her students. Communion refers to the
degree of friendliness or affiliation a teacher conveys in interaction
with his/her students. The loadings on the two dimensions of the QTI can
be placed in a two-dimensional space formed by Agency (vertical) and
Communion (horizontal), see Figure \ref{QTI}. This space is called the
interpersonal circle/circumplex (IPC) and different parts of this space
are characterized by different teacher behavior, e.g.~\enquote{helpful}
or \enquote{uncertain}. The IPC is ``a continuous order with no
beginning or end'' (Gurtman, 2009, p. 2). We call such ordering a
circumplex ordering and the IPC is therefore often called the
interpersonal circumplex. The ordering also implies that scores on the
IPC could be viewed as a circular variable. This circular variable
represents the type of interpersonal behavior that a teacher shows
towards his/her students.\newline \indent Cremers et al. (2018a) explain
the circular nature of the IPC data and analyze them as such using a
circular regression model. The two dimension scores, Agency and
Communion, can be converted to a circular score using the two-argument
arctangent function in \eqref{PredVal}\footnote{The selection of
the origin in circumplex data depends on the scaling of the Agency and Communion
scores. Agency and Communion are measured on a scale from 1 to 5 and for
analysis purposes they are later converted to scale ranging from -1 to 1.  Their
respective 0 scores form the origin. Scaling is however only considered an issue
in those instances where cylindrical data is derived from measurements in
bivariate space.}, where \(A\) represents a score on the Agency
dimension and \(C\) represents a score on the Communion dimension. Note
that when placing a unit circle on Figure \ref{QTI} we see that the
Agency dimension is related to the sine of the circular score and the
Communion dimension is related to the cosine of the circular score.
\begin{equation}\label{PredVal}
\theta          = \text{atan2}\left(A, \: C\right)  =
\left\{{\begin{array}{lcl}
                                                                       \arctan\left(\frac{A}{C}\right) & \text{if}  \quad&C > 0 \\
\arctan\left(\frac{A}{C}\right) + \pi & \text{if}  \quad& C  <  0  \:\: \&\:\: A \geq 0\\
 \arctan\left(\frac{A}{C}\right) - \pi & \text{if}  \quad&C  <  0 \:\:  \&\:\:A  < 0\\
 +\frac{\pi}{2} & \text{if}  \quad& C  =  0  \:\: \&\:\:A > 0\\
 -\frac{\pi}{2} & \text{if}  \quad& C =  0  \:\: \&\:\:A < 0\\
 \text{undefined} & \text{if} \quad& C =  0   \:\: \&\:\:A = 0.
 \end{array}}
\right.
\end{equation} The resulting circular variable \(\theta\) can then be
modelled and takes values in the interval \([0, 2\pi)\) or
\([0^\circ, 360^\circ)\). Note that the round brackets mean that
\(2\pi\) and \(360^\circ\) are not included in the interval since these
represent the same value as 0 as a result of periodicity. \newline
\indent A circular analysis of circumplex data has several benefits: it
is more in line with its theoretical definition and it allows us to
analyse the blend of the two dimensions Agency and Communion instead of
both dimensions separately. This provides us with new insights compared
to a separate analysis of the two dimensions that is standard in the
literature (see e.g. Pennings et al. (2018), Wright et al. (2009) or
Wubbels et al. (2006)). There is however one main drawback: when
two-dimensional data are converted to the circle we lose some
information, namely the length of the two-dimensional vector
\((A, C)^t\), \emph{i.e.}, its Euclidean norm
\(\mid\mid (A, C)^t \mid\mid\). This length represents the strength of
the interpersonal behavior a teacher shows towards his/her students. In
a cylindrical model this strength (the linear outcome) can be modeled
together with the type of interpersonal behavior of a teacher (the
circular outcome). This leads to an improved analysis of interpersonal
circumplex data, over either analyzing the two dimensions separately or
using a circular model, because we take all information, circular and
linear, into account. In the next section we introduce several
cylindrical models that can be used to analyze the teacher data. First
however we will provide descriptives for the teacher data.

\subsection{Data Description}\label{DataDescriptives}

The teacher data was collected between 2010 and 2015 and contains
several repeated measures on the IPC of 161 teachers. Measurements were
obtained using the QTI and taken in different years and classes. For
this paper we only consider one measurement, the first occasion (2010)
and largest class if data for multiple classes were available. This
results in a sample of 151 teachers. In addition to the type of
interpersonal behavior (IPC), the circular outcome, and the strength of
interpersonal behavior (IPC strength), the linear outcome, a teachers'
self-efficacy (\verb|SE|) concerning classroom management is used as
covariate in the analysis. In previous research, in psychology and
education it has been shown that higher self-efficacy is related to the
quality of interpersonal interactions (Locke \& Sadler, 2007;
\VANDER{Want}{Van der}{van der} Want et al., 2018). After listwise
deletion of missings (\(3\) in total, only for the self-efficacy) we
have a sample of 148 teachers. Table \ref{Tableteacherdescriptives}
shows descriptives for the dataset. For the circular variable IPC we
show sample estimates for the circular mean \(\bar{\theta}\) and
concentration \(\hat{\rho}\). The circular concentration lies between 0,
meaning the data is not concentrated at all \emph{i.e.} spread over the
entire circle, and 1, meaning all data is concentrated at a single point
on the circle The population values of these parameters are usually, and
also in this paper, referred to as \(\mu\) (circular location) and
\(\kappa\) (circular concentration). For the linear variables (strength
IPC and SE) we show sample estimates of the linear mean and standard
deviation (sd). Figure \ref{dataplot} is a scatterplot showing the
relation between the linear and circular outcome of the teacher data for
teachers with low SE (below 1 sd below the mean), average SE (between 1
sd below and 1 sd above the mean) and high SE (above 1 sd above the
mean).

\begin{figure}
\centering
\includegraphics[width = \textwidth]{Plots/dataplot.pdf}
\caption{Plot showing the relation between the linear and circular outcome component (in degrees) of the teacher data.}
\label{dataplot}
\end{figure}

\begin{table}[h]
\centering
\caption{Descriptives for the teacher dataset.} 
\begin{tabular}{lrrrl}
  \noalign{\smallskip}\hline\noalign{\smallskip}
Variable & mean/$\bar{\theta}$ & sd/$\hat{\rho}$ & Range & Type \\ \hline\noalign{\smallskip}
IPC &33.22$^\circ$& 0.76 & - & Circular\\
strength IPC & 0.43 & 0.15 & 0.08 - 0.80 & Linear\\
SE & 5.04 & 1.00 & 1.5 - 7.0 & Linear\\
   \hline
\multicolumn{5}{l}{Note: For the circular variable IPC we show sample }\\
\multicolumn{5}{l}{estimates for the circular mean $\bar{\theta}$ and concentration $\hat{\rho}$.}\\
\multicolumn{5}{l}{For the linear variable we show the sample mean,}\\
\multicolumn{5}{l}{standard deviation and range.}
\end{tabular}
\label{Tableteacherdescriptives}
\end{table}

\section{Four Cylindrical Regression Models}\label{Models}

One of the goals of this paper is to show the benefits of cylindrical
methods for the analysis of circumplex data and cylindrical data in
psychology in general. To do so we decide to focus on four cylindrical
models. The models were selected for their relatively low complexity and
the ease with which a regression structure could be incorporated. But
also because they show different ways of modelling the the linear and
circular outcome and thereby illustrate a wider range of cylindrical
models available in the literature. The cylindrical models contain a set
of \(q\) predictors \(\boldsymbol{x} = x_1, \dots, x_q\) and \(p\)
predictors \(\boldsymbol{z} = z_1, \dots, z_q\) for the linear and
circular outcomes, \(Y\) and \(\Theta\), respectively. The first two
models are based on a construction by Mastrantonio et al. (2015), while
the other models are extensions of the models from Abe and Ley (2017)
and Mastrantonio (2018). The four cylindrical models are introduced
separately in the subsections below. However, to provide a more succinct
overview and comparison of the four models, Table \ref{TableModels}
gives an overview of the similarities and differences between the
models.

\begin{sidewaystable}[h]
\centering
\caption{Comparison of the four cylindrical regression models} 
\begin{tabular}{lllll}
  \noalign{\smallskip}\hline\noalign{\smallskip}
\multicolumn{1}{l}{Aspect} & CL-PN & CL-GPN  & Abe-Ley  & GPN-SSN \\ \hline\noalign{\smallskip}
$\Theta$ & &&&\\
$\:\:$Distribution& PN & GPN & Sine-skewed vM & GPN\\
$\:\:$Domain & $[0, 2\pi)$ & $[0, 2\pi)$ & $[0, 2\pi)$ & $[0, 2\pi)$\\
$\:\:$Shape & symmetric, & assymetric, & assymetric, & assymetric, \\
            & unimodal  & multimodal & unimodal   & multimodal \\\hline\noalign{\smallskip}
$Y$& &&&\\
$\:\:$Distribution & Normal & Normal & Weibull & skewed-Normal\\
$\:\:$Domain & $(-\infty, + \infty)$ & $(-\infty, + \infty)$ & $(0, + \infty)$ & $(0, + \infty)$\\
$\:\:$Shape & symmetric, & symmetric, & assymetric, & assymetric, \\
            & unimodal  & unimodal  & unimodal   & unimodal\\\hline\noalign{\smallskip}
$\Theta$-$Y$ dependence &                                   &                                   & & \\
                        & $y$ regressed on                  & $y$ regressed on                  & $\alpha$ and $\kappa$ & multivariate \\
                        & $\sin(\theta)$ and $\cos(\theta)$ & $\sin(\theta)$ and $\cos(\theta)$ & & distribution\\\hline
\multicolumn{5}{l}{Note: PN and GPN refer to the projected normal and general projected normal distribution.}\\
\multicolumn{5}{l}{vM refers to the von-Mises distribution}\\

\end{tabular}
\label{TableModels}
\end{sidewaystable}

\subsection{The Modified Circular-Linear Projected Normal (CL-PN) and Modified Circular-Linear General Projected Normal (CL-GPN) Models}\label{CL-(G)PN}

Following Mastrantonio et al. (2015) we consider two models where the
relation between \(\Theta \in [0, 2\pi)\) and
\(Y\in (-\infty, + \infty)\) and \(q\) covariates is specified as
\begin{equation}\label{circlinlink}
Y = \gamma_0 + \gamma_{cos}*\cos(\Theta)*R + \gamma_{sin}*\sin(\Theta)*R + \gamma_1*x_1 + \dots + \gamma_q*x_q +  \epsilon,
\end{equation} \noindent where the random variable \(R\geq0\) will be
introduced below, the error term \(\epsilon \sim N(0, \sigma^2)\) with
variance \(\sigma^2>0\),
\(\gamma_0, \gamma_{cos}, \gamma_{sin}, \gamma_1, \dots, \gamma_q\) are
the intercept and regression coefficients and \(x_1, \dots, x_q\) are
the \(q\) covariates for the prediction of the linear outcome. We thus
assume a normal distribution for the linear outcome.\newline \indent For
the circular outcome we assume either a projected normal (PN) or a
general projected normal (GPN) distribution. These distributions arise
from a projection of a distribution defined in bivariate space onto the
circle. Figure \ref{projection} represents this projection. In the left
plot of Figure \ref{projection} we see datapoints from the bivariate
variable \(\boldsymbol{S}\) that in the middle plot are projected to
form the circular outcome \(\Theta\) in the right plot. Mathematically
the relation between \(\boldsymbol{S}\) and \(\Theta\) is defined as
follows \begin{equation}\label{projection}
\boldsymbol{S} = \begin{bmatrix} S^{I} \\ S^{II} \end{bmatrix} = R\boldsymbol{u} = \begin{bmatrix} R \cos (\Theta) \\  R\sin (\Theta) \end{bmatrix},
\end{equation} \noindent where \(R = \mid\mid \boldsymbol{S} \mid\mid\),
the Euclidean norm of \(\boldsymbol{S}\); the lines connecting the
bivariate datapoints to the origin in the middle plot. We call
\(\boldsymbol{S}\) the augmented representation of the circular outcome,
it is a variable that in contrast to \(\Theta\) is not observed and thus
considered latent or auxiliary. This then means that we do not model
\(\Theta\) directly but indirectly through \(\boldsymbol{S}\). \newline
\indent For both the PN and GPN distribution the circular location
parameter \(\mu\in [0, 2\pi)\) is modeled as
\(\hat{\mu}_i = \mbox{atan2}(\hat{\mu}_i^{II}, \hat{\mu}_i^{I}) = \mbox{atan2}(\boldsymbol{\beta}^{II}\boldsymbol{z}_i, \boldsymbol{\beta}^{I}\boldsymbol{z}_i)\)\footnote{Note that for the CL-GPN
model the circular location parameter also depends on the variance-covariance
matrix and the circular predicted values should be computed using numerical
integretion or Monte Carlo methods because a closed form expression for the mean
direction is not available.} where \(\boldsymbol{\beta}^{I}\) and
\(\boldsymbol{\beta}^{II}\) are vectors with intercepts and regression
coefficients for the prediction of \(\boldsymbol{S}\) and
\(\boldsymbol{z}_i\) is a vector with predictor values for each
individual \(i \in 1, \dots, n\) where \(n\) is the sample size. Note
that as a result of the augmented representation of the circular outcome
we have two sets of regression coefficients and intercepts, one for each
bivariate component of \(\boldsymbol{S}\). This leads to problems when
we want to interpret the effect of a covariate on the circle. A circular
regression line is shown in Figure \ref{circregline}, with covariate
values on the x-axis and the predicted circular outcome on the y-axis.
As can be seen it is of a non-linear character meaning that the effect
of a covariate is different at different values of the covariate. A
circular regression line is usually described by the slope at the
inflection point, the point at which the slope of the regression line
starts flattening off (indicated with a square in Figure
\ref{circregline}). By default, the parameters from the PN and GPN model
do not directly describe this inflection point. For the PN distribution
however, Cremers et al. (2018b) solved this interpretation problem and
introduce new circular regression coefficients. They introduce a new
parameter \(b_c\) that describes the slope at the inflection point of
the regression line. For the GPN distrbution the interpretation problem
however remains.\newline \indent The main difference between the PN and
GPN distribution lies in the definition of their covariance matrix. For
the PN distribution this is an identity matrix, causing the distribution
to be unimodal and symmetric, whereas for the GPN distribution
\(\boldsymbol{\Sigma} = \begin{bmatrix} \tau^2 + \xi^2 & \xi\\ \xi & 1 \end{bmatrix}\),
allowing for multimodality and assymetry/skewness.\newline \indent For
the teacher dataset the predicted linear outcome, strengths of
interpersonal behavior, in the CL-PN and CL-GPN model is the following:
\[\hat{y}_i = \gamma_0 + \gamma_{cos}\cos(\theta_i)r_i +
\gamma_{sin}\sin(\theta_i)r_i + \gamma_1\text{SE}_i.\] The predicted
circular outcome, type of interpersonal behavior, equals:
\[\hat{\theta}_i = \mu_i =
\mbox{atan2} (\beta_0^{II} + \beta_1^{II}\text{SE}_i, \beta_0^{I} +
\beta_1^{I}\text{SE}_i).\] \noindent where \(\text{SE}_i\) is the
self-efficacy score of one individual. The CL-PN and CL-GPN models thus
allow us to assess the average type and strength of interpersonal
behavior through the parameters \(\beta_{0}^{I}\), \(\beta_{0}^{II}\)
and \(\gamma_0\) as well as the effect of self-efficacy on type and
strength of teacher behavior through the parameters, \(\beta_{1}^{I}\),
\(\beta_{1}^{II}\) and \(\gamma_1\). In addition, because the type and
strength of interpersonal behavior are modelled together via the
regression in \eqref{circlinlink} we can assess the effect of the type
of interpersonal behavior on the strength through the parameters
\(\gamma_{sin}\) and \(\gamma_{cos}\). In the teacher data these are the
regression coefficients for the effect of the sine and cosine of the
type of behavior which are related to the scores on the Agency an
Communion dimensions respectively.\newline \indent Both the CL-PN and
CL-GPN models are estimated using Markov Chain Monte Carlo (MCMC)
methods based on Nuñez-Antonio, Gutiérrez-Peña, and Escarela (2011),
Wang and Gelfand (2013) and Hernandez-Stumpfhauser, Breidt, and Woerd
(2016) for the regression of the circular outcome. A detailed
description of the Bayesian estimation and MCMC samplers can be found in
the Supplementary Material.

\begin{figure}
\centering
\includegraphics[width = \textwidth]{Plots/plotprojecting.pdf}
\caption{Plot showing the projection of datapoints in bivariate space, $\boldsymbol{S}$, (left) to the circle (right). The lines connecting the bivariate datapoints to the circular datapoints represent the euclidean norm of the bivariate datapoints, the random variable $R$.}
\label{projection}
\end{figure}

\begin{figure}[]
  \includegraphics[width = \textwidth]{Plots/circregline.pdf}
  \caption{Circular regression line for the relation between a covariate and a circular outcome with the data the regression line was fit to. The square indicates the inflection point of the regression live.}
  \label{circregline}
\end{figure}

\subsection{The Modified Abe-Ley Model}\label{WeiSSVM}

This model is an extension of the cylindrical model introduced in Abe
and Ley (2017) to the regression context. It concerns a combination of a
Weibull distribution, with scale parameter \(\nu>0\) and shape parameter
\(\alpha\), for the linear outcome and a sine-skewed von Mises
distribution, with location parameter \(\mu\in [0, 2\pi)\),
concentration parameter \(\kappa>0\) and skewness
\(\lambda \in [-1,1]\), for the circular outcome. In contrast to the
CL-PN and CL-GPN models, the linear outcome \(Y\) is in this model
defined only on the positive real half-line \([0, + \infty)\) and thus
can not be negative.\newline \indent In this model we predict the linear
scale parameter and circular location parameter, both of which we can
express in terms of covariates:
\(\hat{\nu}_i = \exp(\boldsymbol{x}_i^t\boldsymbol{\gamma}) > 0\) and
\(\hat{\mu}_i = \beta_0 + 2\tan^{-1}(\boldsymbol{z}_i^t\boldsymbol{\beta})\).
The parameter \(\boldsymbol{\gamma}\) is a vector of \(q\) regression
coefficients \(\gamma_j \in (-\infty, +\infty)\) for the prediction of
\(y\) where \(j = 0, \dots, q\) and \(\gamma_0\) is the intercept. The
parameter \(\beta_0 \in [0, 2\pi)\) is the intercept and
\(\boldsymbol{\beta}\) is a vector of \(p\) regression coefficients
\(\beta_j \in (-\infty, +\infty)\) for the prediction of \(\theta\)
where \(j = 1, \dots, p\). The vector \(\boldsymbol{x}_i\) is a vector
of predictor values for the prediction of \(y\) and \(\boldsymbol{z}_i\)
is a vector of predictor values for the prediction of
\(\theta\).\newline \indent For the teacher data, the predicted values
for the circular outcome in the Abe-Ley model are:
\[\hat{\theta}_{i} = \hat{\mu}_i = \beta_0 + 2 *
\tan^{-1}(\beta_1\text{SE}_i).\] We do not directly predict the linear
outcome. The conditional distribution for the linear outcome is Weibull,
meaning that we can use methods from survival analsis to interpret the
effect of a predictor. In survival analysis a \enquote{survival}
function is used in which time is plotted against the probability of
survival of subjects suffering from a specific medical condition. In the
teacher data we can thus compute the probability of a teacher having a
specific strength on the IPC. This probability is computed using the
\enquote{survival-function} defined as \[\exp(-\alpha
y_i^{\hat{\nu}_i(1-\tanh(\kappa)\cos(\theta_i - \hat{\mu}_i))^{1/\alpha}}),\]
with \(\hat{\nu}_i = \exp(\gamma_0 + \gamma_1\mbox{SE}_i)\). From the
survival function we also see that the circular concentration parameter
\(\kappa\) and the linear shape parameter \(\alpha\) regulate the
circular-linear dependence in the Abe-Ley model. The Abe-Ley model thus
allows us to assess the average type and strength of interpersonal
behavior through the parameters \(\beta_{0}\) and \(\gamma_0\) as well
as the effect of self-efficacy on type and strength of teacher behavior
through the parameters, \(\beta_{1}\) and \(\gamma_1\).\newline
\indent We can use numerical optimization (Nelder-Mead) to find
solutions for the maximum likelihood (ML) estimates for the parameters
of the model.

\subsection{Modified Joint Projected and Skew Normal Model (GPN-SSN)}\label{CL-GPN_multivariate}

This model is an extension of the cylindrical model introduced by
Mastrantonio (2018) to the regression context. The model contains \(m\)
independent circular outcomes and \(w\) independent linear outcomes. The
circular outcomes
\(\boldsymbol{\Theta} = (\boldsymbol{\Theta}_1, \dots,  \boldsymbol{\Theta}_m)\)
are modelled together by a multivariate GPN distribution. The linear
outcomes
\(\boldsymbol{Y} = (\boldsymbol{Y}_1, \dots,  \boldsymbol{Y}_w)\) are
modelled together by a multivariate skew normal distribution (Sahu, Dey,
\& Branco, 2003). Thus also in this model the linear outcome is defined
constrained to be positive. Because the GPN distribution is modelled
using a so-called augmented representation (see also the description of
the CL-PN and CL-GPN models) it is convenient to use a similar tactic
for modelling the multivariate skew normal distribution. As in
Mastrantonio (2018) dependence between the linear and circular outcome
is created by modelling the augmented representations of
\(\boldsymbol{\Theta}\) and \(\boldsymbol{Y}\) together in a \(2m + w\)
dimensional normal distribution.\newline \indent This means that we have
a shared mean vector and variance-covariance matrix for the linear and
circular outcome(s), much like having multiple outcomes in a MANOVA
(multivariate analysis of variance) model. In our regression extension
of the GPN-SSN model we have \(i = 1, \dots, n\) observations of \(m\)
circular outcomes, \(w\) linear outcomes and \(g\) covariates. The mean
vector then becomes
\(\boldsymbol{M}_i = \boldsymbol{B}^t\boldsymbol{x}_i\) where
\(\boldsymbol{B}\) is a \((g + 1) \times (2m + w)\) matrix with
regression coefficients and intercepts and \(\boldsymbol{x}_i\) is a
\(g + 1\) dimensional vector containing the value 1 to estimate an
intercept and the \(g\) covariate values. This means that in contrast to
the other three models, we have to use the same set of predictors for
the circular and linear outcome.\newline \indent For the teacher data,
\(\boldsymbol{B} = \begin{bmatrix} \beta_{0_{s^{I}}} & \beta_{0_{s^{II}}} & \beta_{0_{y}}\\ \beta_{1_{s^{I}}} & \beta_{1_{s^{II}}} & \beta_{1_{y}} \end{bmatrix}\).
The predicted
circular\footnote{Note that for the  GPN-SSN model the predicted circular outcome also depends on the variance-covariance matrix and the circular predicted values should be computed using numerical integretion or Monte Carlo methods because a closed form expression for the mean direction is not available.}
and linear outcomes in the GPN-SSN model are
\[\hat{\theta_i} = \mbox{atan2}(\beta_{0_{s^{II}}} +
\beta_{1_{s^{II}}}\text{SE}_i,\beta_{0_{s^{I}}} +
\beta_{1_{s^{I}}}\text{SE}_i),\]

\noindent and

\[\hat{y_i} = \beta_{0_{y_i}} +
\beta_{1_{y_i}}\text{SE}_i.\]

\noindent The GPN-SSN model thus allows us to assess the average type
and strength of interpersonal behavior through the parameters
\(\beta_{0_{s^{I}}}\), \(\beta_{0_{s^{II}}}\) and \(\beta_{0_{y}}\) as
well as the effect of self-efficacy on type and strength of teacher
behavior through the parameters, \(\beta_{1_{s^{I}}}\),
\(\beta_{1_{s^{II}}}\) and \(\beta_{1_{y}}\). In addition, because the
type and strength of interpersonal behavior are modelled together using
a multivariate normal distribution we can through its
variance-covariance matrix also assess the dependence between the type
and strength of interpersonal behavior. \newline \indent We estimate the
model using MCMC methods. A detailed description of these methods is
given in the Supplementary Material.

\subsection{Model Fit Criterion}\label{Modelfit}

For the four cylindrical models we focus on their out-of-sample
predictive performance to determine the fit of the model. To do so we
use k-fold cross-validation and split our data into 10 folds. Each of
these folds (10 \(\%\) of the sample) is used once as a holdout set and
9 times as part of a training set. The analysis will thus be performed
10 times, each time on a different training set.\newline \indent A
proper criterion to compare out-of-sample predictive performance is the
Predictive Log Scoring Loss (PLSL) (Gneiting \& Raftery, 2007). The
lower the value of this criterion, the better the predictive performance
of the model. Using ML estimates this criterion can be computed as
follows: \begin{equation}\label{PLSLML}
PLSL = -2 \sum_{i = 1}^{M}\log l(x_i \mid \hat{\boldsymbol{\vartheta}}),\nonumber
\end{equation} \noindent where \(l\) is the model likelihood, \(M\) is
the sample size of the holdout set, \(x_i\) is the \(i^{th}\) datapoint
from the holdout set and \(\hat{\boldsymbol{\vartheta}}\) are the ML
estimates of the model parameters. Using posterior samples the criterion
is similar to the log pointwise predictive density (lppd) (Gelman et
al., 2014, p. 169) and can be computed as:
\begin{equation}\label{PLSLBayes}
PLSL = -2 \frac{1}{B} \sum_{j = 1}^{B}\sum_{i = 1}^{M} \log l(x_i \mid \boldsymbol{\vartheta}^{(j)}),\nonumber
\end{equation} \noindent where \(B\) is the amount of posterior samples
and \(\boldsymbol{\vartheta}^{(j)}\) are the posterior estimates of the
model parameters for the \(j^{th}\) iteration. Because the joint density
and thus also the likelihood for the modified GPN-SSN model is not
available in closed form (Mastrantonio, 2018) we compute the PLSL for
the circular and linear outcome separately for all models. Note that
although we fit the CL-PN, CL-GPN and GPN-SSN models using Bayesian
statistics, we do not take prior information into account when assessing
model fit with the PLSL. According to Gelman et al. (2014) this is not
necessary since we are assessing the fit of a model to data, the holdout
set, only. They argue that the prior in such case is only of interest
for estimating the parameters of the model but not for determining the
predictive accuracy.\newline \indent For each of the four cylindrical
models and for each of the 10 cross-validation analyses we can then
compute a PLSL for the circular and linear outcome by using the
conditional log-likelihoods of the respective outcome (see Supplementary
Material for a definition of the loglikelihoods). To evaluate the
predictive performance we average across the PLSL criteria of the
cross-validation analyses. We also assess the cross-validation
variability by means of the standard deviations of the PLSL criteria.

\section{Data Analysis}\label{DataAnalysis}

In this section we analyze the teacher data with the help of the four
cylindrical models from the previous section. We will present the
results, posterior estimates and their interpretation for all four
models.

\subsection{Results \& Analysis}\label{DataResults}

In the Supplementary Material we have described the starting values for
the MCMC procedures for the CL-PN, CL-GPN and GPN-SSN models, hence it
remains to specify the starting values for the maximum likelihood based
Abe-Ley model:
\(\eta_0 = 0.9, \eta_1 = 0.9, \nu_0 = 0.9, \nu_1 = 0.9, \kappa = 0.9, \alpha = 0.9, \lambda = 0\).
The initial number of iterations for the three MCMC samplers was set to
2000. After convergence checks via traceplots we concluded that some of
the parameters of the GPN-SSN model did not converge. Therefore we set
the number of iterations of the MCMC models to 20,000 and subtracted a
burn-in of 5000 to reach convergence (the Geweke diagnostics show
absolute z-scores over 1.96 in 6\% of the estimated parameters). Note
that we choose the same number of iterations for all three models models
estimated using MCMC prodedures to make their comparison via the PLSL as
fair as possible. Lastly, the predictor \verb|SE| was centered before
inclusion in the analysis as this allows the intercepts to bear the
classical meaning of average behavior.\newline \indent Tables
\ref{tab:estCLGPN}, \ref{tab:estAL} and \ref{tab:estCLGPNM} show the
results for the four cylindrical models that were fit to the teacher
data. For the models estimated using MCMC methods, CL-PN, CL-GPN and
GPN-SSN we show descriptives of the posterior of the estimated
parameters (posterior mode and lower and upper bound of the 95\% highest
posterior density (HPD) interval). For the Abe-Ley model we show the
maximum likelihood estimates of the parameters. To compare the results
of the four models we focus on the following aspects: the estimated
average scores (intercept) on the type and strength of interpersonal
behavior (1), the effect of self-efficacy on the type and strength of
interpersonal behavior (2), the dependence between the type and strength
of interpersonal behavior (3) and the model fit (4).

\begin{table}

\caption{\label{tab:estCLGPN}Results, cross-validation mean and standard deviation, for the modified CL-PN and CL-GPN models}
\centering
\begin{tabular}[t]{lllllll}
\toprule
\multicolumn{1}{c}{Parameter} & \multicolumn{3}{c}{CL-PN} & \multicolumn{3}{c}{CL-GPN} \\
\cmidrule(l{2pt}r{2pt}){1-1} \cmidrule(l{2pt}r{2pt}){2-4} \cmidrule(l{2pt}r{2pt}){5-7}
  & Mode & HPD LB & HPD UB & Mode & HPD LB & HPD UB\\
\midrule
$\beta_0^{I}$ & 1.76 (0.09) & 1.50 (0.07) & 2.03 (0.09) & 2.43 (0.12) & 1.91 (0.10) & 3.05 (0.17)\\
$\beta_1^{I}$ & 0.65 (0.07) & 0.42 (0.06) & 0.90 (0.08) & 0.84 (0.11) & 0.45 (0.09) & 1.29 (0.15)\\
$\beta_0^{II}$ & 1.15 (0.05) & 0.92 (0.04) & 1.37 (0.04) & 1.47 (0.05) & 1.16 (0.04) & 1.78 (0.05)\\
$\beta_1^{II}$ & 0.58 (0.03) & 0.38 (0.04) & 0.79 (0.04) & 0.70 (0.06) & 0.47 (0.05) & 0.96 (0.08)\\
$\gamma_0$ & 0.38 (0.01) & 0.31 (0.01) & 0.44 (0.01) & 0.37 (0.01) & 0.31 (0.01) & 0.42 (0.01)\\
\addlinespace
$\gamma_{cos}$ & 0.04 (0.00) & 0.01 (0.00) & 0.06 (0.00) & 0.03 (0.00) & 0.01 (0.00) & 0.04 (0.00)\\
$\gamma_{sin}$ & -0.01 (0.00) & -0.04 (0.00) & 0.02 (0.00) & -0.00 (0.00) & -0.03 (0.00) & 0.03 (0.00)\\
$\gamma_1$ & 0.03 (0.01) & -0.00 (0.00) & 0.07 (0.01) & 0.03 (0.00) & -0.00 (0.00) & 0.06 (0.00)\\
$\sigma$ & 0.14 (0.00) & 0.12 (0.00) & 0.16 (0.00) & 0.14 (0.00) & 0.12 (0.00) & 0.16 (0.00)\\
$\sum_{1,1}$ & NA (NA) & NA (NA) & NA (NA) & 3.04 (0.29) & 1.85 (0.13) & 5.00 (0.41)\\
\addlinespace
$\sum_{1,2}$ & NA (NA) & NA (NA) & NA (NA) & 0.47 (0.12) & 0.12 (0.12) & 0.80 (0.10)\\
$\sum_{2,2}$ & NA (NA) & NA (NA) & NA (NA) & 1.00 (0.00) & 1.00 (0.00) & 1.00 (0.00)\\
\bottomrule
\multicolumn{7}{l}{Note: $\beta_0^{I}$, $\beta_0^{II}$ and $\gamma_0$ inform us about the type and strength of interpersonal behavior}\\
\multicolumn{7}{l}{at the average self-efficacy. $\beta_1^{I}$, $\beta_1^{II}$ and $\gamma_1$ inform us about the effect of self-efficacy on the}\\
\multicolumn{7}{l}{type and strength of interpersonal behavior. $\gamma_{cos}$ and $\gamma_{sin}$ inform us about the dependence}\\
\multicolumn{7}{l}{ between the type and strength of interpersonal behavior. $\sum_{1,1}$, $\sum_{1,2}$ and $\sum_{2,2}$ are are}\\
\multicolumn{7}{l}{elements of the variance-covariance matrix of the type of interpersonal behavior in the}\\
\multicolumn{7}{l}{CL-GPN model and $\sigma$ is the error standard deviation of the strength of interpersonal behavior.}\\

\end{tabular}
\end{table}

\begin{table}

\caption{\label{tab:estAL}Results, cross-validation mean and standard deviation (SD), for the modified Abe-Ley model}
\centering
\begin{tabular}[t]{llllllll}
\toprule
& $\beta_0$ & $\beta_1$ & $\gamma_0$  & $\gamma_1$ & $\alpha$ & $\kappa$ & $\lambda$\\
Mean & 0.36 & -0.03 & 1.17 & 0.04 & 3.66 & 1.51 & 0.70 \\
SD & 0.02) & 0.01 & 0.02 & 0.02 & 0.12 & 0.08 & 0.05\\
\bottomrule
\multicolumn{8}{l}{Note: $\beta_0$  and $\gamma_0$ inform us about the type and strength }\\
\multicolumn{8}{l}{of interpersonal behavior at the average self-efficacy. $\beta_1$ }\\
\multicolumn{8}{l}{and $\gamma_1$ inform us about the effect of self-efficacy on the}\\
\multicolumn{8}{l}{type and strength of interpersonal behavior. $\alpha$ is the } \\
\multicolumn{8}{l}{shape parameter of the distribution of the strength }\\
\multicolumn{8}{l}{of interpersonal bahavior. $\kappa$ and $\lambda$ are the concentration}\\
\multicolumn{8}{l}{and skewness parameters for the distribution of the type}\\
\multicolumn{8}{l}{of interpersonal behavior.}\\
\end{tabular}
\end{table}

\begin{table}
\caption{\label{tab:estCLGPNM}Results, cross-validation mean and standard deviation, for the GPN-SSN model}
\centering
\begin{tabular}[t]{lllllll}
\toprule
\multicolumn{1}{c}{Parameter} & \multicolumn{3}{c}{Unconstrained} & \multicolumn{3}{c}{Constrained} \\
\cmidrule(l{2pt}r{2pt}){1-1} \cmidrule(l{2pt}r{2pt}){2-4} \cmidrule(l{2pt}r{2pt}){5-7}
  & Mode & HPD LB & HPD UB & Mode & HPD LB & HPD UB\\
\midrule
$\beta_{0_s^{I}}$ & 0.30 (0.01) & 0.26 (0.01) & 0.34 (0.01) & 2.11 (0.11) & 1.75 (0.09) & 2.50 (0.11)\\
$\beta_{0_s^{II}}$ & 0.19 (0.00) & 0.17 (0.01) & 0.21 (0.00) & 1.34 (0.06) & 1.10 (0.05) & 1.57 (0.06)\\
$\beta_{0_y}$ & 0.33 (0.01) & 0.30 (0.30) & 0.36 (0.01) & 0.33 (0.01) & 0.30 (0.01) & 0.36 (0.01)\\
\addlinespace
$\beta_{1_s^{I}}$ & 0.09 (0.01) & 0.05 (0.01) & 0.13 (0.01) & 0.60 (0.06) & 0.33 (0.05) & 0.90 (0.06)\\
$\beta_{1_s^{II}}$ & 0.07 (0.00) & 0.04 (0.00) & 0.09 (0.01) & 0.48 (0.03) & 0.30 (0.04) & 0.66 (0.04)\\
$\beta_{1_y}$ & 0.09 (0.01) & 0.06 (0.06) & 0.12 (0.01) & 0.09 (0.01) & 0.06 (0.01) & 0.12 (0.01)\\
\addlinespace
$\sum_{s_{1,1}}$ & 0.05 (0.00) & 0.04 (0.00) & 0.06 (0.00) & 2.44 (0.15) & 1.72 (0.07) & 3.46 (0.14)\\
$\sum_{s_{2,2}}$ & 0.02 (0.00) & 0.02 (0.00) & 0.03 (0.00) & 1.00 (0.00) & 1.00 (0.00) & 1.00 (0.00)\\
$\sum_{y_{3,3}}$ & 0.03 (0.00) & 0.02 (0.02) & 0.04 (0.00) & 0.03 (0.00) & 0.02 (0.00) & 0.04 (0.00)\\
$\sum_{s_{1,2}}$ & 0.00 (0.00) & -0.00 (0.00) & 0.01 (0.00) & 0.08 (0.06) & -0.20 (0.06) & 0.34 (0.06)\\
$\sum_{sy_{1,3}}$ & 0.03 (0.00) & 0.02 (0.00) & 0.04 (0.00) & 0.23 (0.01) & 0.17 (0.00) & 0.32 (0.01)\\
$\sum_{sy_{2,3}}$ & 0.01 (0.00) & 0.01 (0.01) & 0.02 (0.00) & 0.09 (0.01) & 0.06 (0.01) & 0.12 (0.01)\\
$\lambda$ & 0.16 (0.01) & 0.14 (0.01) & 0.18 (0.01) & 0.16 (0.01) & 0.14 (0.01) & 0.18 (0.01)\\
\bottomrule
\multicolumn{7}{l}{Note: $\beta_{0_s^{I}}$, $\beta_{0_s^{II}}$ and $\beta_{0_y}$ inform us about the type and strength of interpersonal behavior}\\
\multicolumn{7}{l}{at the average self-efficacy. $\beta_{1_s^{I}}$, $\beta_{1_s^{II}}$ and $\beta_{1_y}$ inform us about the effect of self-efficacy }\\
\multicolumn{7}{l}{on the type and strength of interpersonal behavior. $\sum_{s_{1,1}}$, $\sum_{s_{1,2}}$, $\sum_{s_{2,2}}$, $\sum_{y_{3,3}}$, $\sum_{sy_{1,3}}$,  and $\sum_{sy_{2,3}}$ }\\
\multicolumn{7}{l}{are elements of the variance-covariance matrix of which $\sum_{sy_{1,3}}$,  and $\sum_{sy_{2,3}}$ inform us about}\\
\multicolumn{7}{l}{the dependence between the type and strength of interpersonal behavior.}\\
\multicolumn{7}{l}{$\lambda$ is the skewness parameter of the distribution of the strengths of interpersonal behavior.}\\
\end{tabular}
\end{table}

\subsubsection{Average type and strength of interpersonal behavior}

The parameters \(\gamma_0\) in the CL-PN, CL-GPN and Abe-Ley model and
the parameter \(\beta_{0_y}\) in the GPN-SSN model inform us about the
strength of interpersonal behavior at the average self-efficacy. For the
CL-PN, CL-GPN and GPN-SSN model the parameters are estimated at 0.38,
0.37 and 0.30 respectively and are a direct prediction of the strength
of interpersonal behavior at the average self-efficacy. The estimate for
the GPN-SSN model is notably lower and likely to be caused by its skewed
distribution for the strengths of interpersonal behavior. In the Abe-Ley
model, \(\gamma_0\) influences the shape parameter of the distribution
of the strength of interpersonal behavior and does not directly estimate
the average strength. Instead we can use the survival function to say
something about the probability of having a certain strength of
interpersonal behavior. Figure \ref{reglineweib} shows this function for
several values of self-efficacy. We look at the survival function at
average values of self-efficacy. Note that this function is the average
of all survival functions for observations that fall within 1 standard
deviation of the mean. The survival function indicates that the
probability of having a low strength of interpersonal behavior is higher
than having a high strength. We however can not make any direct
statement about the estimated strength using the Abe-Ley model.\newline
\indent The parameters \(\beta_0^{I}\), \(\beta_0^{II}\), \(\beta_0\),
\(\beta_{0_{s^{I}}}\) and \(\beta_{0_{s^{II}}}\) inform us about the
type of interpersonal behavior at the average self-efficacy for the
CL-PN, CL-GPN, Abe-Ley and GPN-SSN model respectively. For the CL-PN,
CL-GPN and GPN-SSN model we need to combine the estimates for the
underlying bivariate components \(\{I, II\}\) into one circular estimate
using the double arctangent
function\footnote{\(\mbox{atan2}(\beta_0^{II}, \beta_0^{I})\) or
\(\mbox{atan2}(\beta_{0_{s^{II}}}, \beta_{0_{s^{I}}})\)}. Table
\ref{tab:means} shows that these circular estimates are similar for the
three models at 32.29\(^\circ\), 33.70\(^\circ\) and 35.53\(^\circ\). In
the Abe-Ley model the type of interpersonal behavior at the average
self-efficacy is estimated at 0.36 radians or 20.63\(^\circ\).

\begin{figure}
\centering
\includegraphics[width = \textwidth]{Plots/survivaldiffSE.pdf}
\caption{Plot showing the probability of having a particular strength of interpersonal behavior (survival plot) for the minimum, mean and maximum self-efficacy in the data.}
\label{reglineweib}
\end{figure}

\begin{table}
\caption{\label{tab:means}Posterior estimates (in degrees) for the circular mean (at SE = 0) in the CL-PN, CL-GPN and GPN-SSN models}
\centering
\begin{tabular}[t]{lrrr}
\toprule
  & Mode & HPD LB & HPD UB\\
  \midrule
CL-PN & 32.29 & 24.81 & 39.71\\
CL-GPN & 33.70 & 26.72 & 41.15\\
GPN-SSN & 35.53 & 28.40 & 43.30\\
\bottomrule
\multicolumn{4}{l}{Note that these means are based on}\\
\multicolumn{4}{l}{their posterior predictive distribution }\\
\multicolumn{4}{l}{following (Wang and Gelfand, 2013)}\\
\end{tabular}
\end{table}

\subsubsection{The effect of self-efficacy}

The parameters \(\gamma_1\) in the CL-PN, CL-GPN, Abe-Ley model and
\(\beta_{1_y}\) in the GPN-SSN model inform us about the effect of
self-efficacy on the strength of interpersonal behavior. For the CL-PN,
CL-GPN and GPN-SSN model the parameters are estimated at 0.03, 0.03 and
0.09 respectively and are a direct estimate of the effect of
self-efficacy on the strength of interpersonal behavior, \emph{i.e.} an
increase of 1 unit in self-efficacy leads to an increase of 0.09 units
in the strength of interpersonal behavior according to the GPN-SSN
model. These estimates are however quite small and only different from 0
(the HPD interval does not contain 0) in the GPN-SSN model. It is hard
to say which of the three models, CL-PN, CL-GPN or GPN-SSN, to use to
base our conclusions on. The model CL-GPN and CL-PN fit the linear
outcome best according to the model fit in Table \ref{tab:ModelFit}. In
these models the linear outcome has a symmetric distribution whereas in
the GPN-SSN the distribution of the linear outcome is skewed. However,
the effect of self-efficacy is different from 0 only in the GPN-SSN
model which does not seem to match with its lower model fit.\newline
\indent In the Abe-Ley model, \(\gamma_1\) influences the shape
parameter of the distribution of the strength of interpersonal behavior
and does not directly estimate the effect of self-efficacy. Instead we
can use the survival function to say something about the probability of
having a certain strength of interpersonal behavior for different values
of self-efficacy. Figure \ref{reglineweib} shows this function for low,
average and high values of self-efficacy (as defined in Figure
\ref{dataplot}). This function indicates that the effect of
self-efficacy on the strength of interpersonal behavior is not linear.
The probability of having a higher strength of interpersonal behavior is
highest for low self-efficacy and lowest for average
self-efficacy.\newline \indent The parameters \(\beta_1^{I}\),
\(\beta_1^{II}\), \(\beta_1\), \(\beta_{1_{s^{I}}}\) and
\(\beta_{1_{s^{II}}}\) inform us about the effect of self-efficacy on
the type of interpersonal behavior in the CL-PN, CL-GPN, Abe-Ley and
GPN-SSN model respectively. For the CL-PN and Abe-Ley model we have
drawn the circular regression lines for this effect in Figure
\ref{regline} (see the description of the CL-PN and CL-GPN models). For
the CL-PN model the inflection point is indicated with a square in
Figure \ref{regline}. The inflection point for the Abe-Ley model falls
outside the bounds of the plot and is therefore not displayed. The slope
at the inflection point, \(b_c\), for the CL-PN model is computed by
using methods from Cremers et al. (2018b) and is equal to 1.67 (-24.66,
29.33)\footnote{Note that this is a linear approximation
to the circular regression line representing the slope at a specific point.
Therefore it is possible for the HPD interval to be wider than $2\pi$. In this
case the interval is much wider and covers 0, indicating there is no evidence
for an effect.} The parameter \(\beta_1\) is the slope at the inflection
point for the Abe-Ley model and is equal to -0.03. Even though these
slopes are quite different, the regression lines in Figure \ref{regline}
are quite similar in the data range. Both the regression line of the
Abe-Ley model and the CL-PN model show slopes that are not very steep in
the range of the data indicating that the effect of self-efficacy on the
type of interpersonal behavior is not large. \newline \indent In the
CL-GPN and GPN-SSN model we cannot compute circular regression
coefficients due to the fact that not only the mean vector of the GPN
distribution but also the covariance matrix influences the predicted
value on the circle. Instead, we will compute posterior predictive
distributions for the predicted circular outcome of individuals scoring
the minimum, maximum and median self-efficacy. The modes and 95\% HPD
intervals of these posterior predictive distributions are
\(\hat{\theta}_{SE_{min}} = 215.74^\circ (147.36^\circ, \: 44.49^\circ)\),
\(\hat{\theta}_{SE_{median}} = 25.93^\circ (337.02^\circ, \: 138.59^\circ)\),
\(\hat{\theta}_{SE_{max}} = 30.86^\circ (8.63^\circ, \: 72.19^\circ)\)
for the CL-GPN model. Note that we display the modes and HPD intervals
for the posterior predictive distributions on the interval
\([0^\circ, 360^\circ)\) and that \(44.49^\circ = 404.49^\circ\) due to
the periodicity of a circular variable. The posterior mode estimate of
\(215.74^\circ\) thus lies within its HPD interval
\((147.36^\circ, \: 44.49^\circ)\). For the GPN-SSN model the modes and
95\% HPD intervals of the posterior predictive distributiona are
\(\hat{\theta}_{SE_{min}} = 206.87^\circ (117.12^\circ, \: 72.02^\circ)\),
\(\hat{\theta}_{SE_{median}} = 24.68^\circ (334.73^\circ, \: 128.27^\circ)\),
\(\hat{\theta}_{SE_{max}} = 29.81^\circ (0.74^\circ, \: 80.61^\circ)\).
For both the CL-GPN and GPN-SSN model the HPD intervals of the mode of
the posterior predictive intervals of individuals scoring the minimum,
median and maximum self-efficacy overlap. This indicates that the effect
of self-efficacy, if there is any, on the type of interpersonal behavior
a teacher shows is not expected to be strong. Had the HPD intervals not
overlapped we could have concluded that as the self-efficacy increases,
the score of the teacher on the IPC moves counterclockwise.\newline

\begin{figure}
\centering
\includegraphics[width = \textwidth]{Plots/reglinediffSE.pdf}
\caption{Plot showing circular regression lines for the effect of self-efficacy
as predicted by the Abe-Ley model (solid line) and CL-PN model (dashed line). The
black square indicates the inflection point of the circular regression line for
the CL-PN model.}
\label{regline}
\end{figure}

\subsubsection{Dependence between type and strength of interpersonal behavior.}

The relation between the type and strength of interpersonal behavior in
the CL-PN and CL-GPN model, is described by the parameters
\(\gamma_{\cos}\) and \(\gamma_{\sin}\). The HPD interval of
\(\gamma_{\cos}\) does not include 0 for both the CL-PN and CL-GPN
models, meaning that the cosine component of the type of interpersonal
behavior has an effect on the strength of interpersonal
behavior.\newline \indent In the teacher data the sine and cosine
components have a substantive meaning. This is illustrated in Figure
\ref{QTI}. In a unit circle the horizontal axis (Communion) represents
the cosine of and the vertical axis (Agency) represents the sine of an
angle. For the teacher data this means that the Communion (cosine)
dimension of the IPC positively effects the strength of a teachers' type
of interpersonal behavior, in plain words: teachers exhibiting
interpersonal behavior types with higher communion scores (e.g.,
\enquote{helpful} and \enquote{understanding} in Figure 2) are stronger
in their interpersonal behavior.\newline  \indent In the GPN-SSN model
the dependence between the type and strengths of interpersonal behavior
is modelled through the covariances between the linear outcome and the
sine and cosine of the circular outcome \(\sum_{sy_{2,3}}\) and
\(\sum_{sy_{1,3}}\). Both covariances, \(\sum_{sy_{2,3}} = 0.09\) and
\(\sum_{sy_{1,3}} = 0.23\), are different from zero, but the one of the
cosine component, and thus the correlation with the Communion dimension,
is larger. This mean that teachers scoring both high on Communion and
Agency show stronger behavior. Together with the results from the CL-PN
and CL-GPN models in the previous paragraph this translates to the
conclusion that teachers with the strongest interpersonal behavior have
a type of interpersonal behavior between 0\(^\circ\) and 90\(^\circ\).
To get these scores on the circle both the Agency and the Communion
score of a Teacher have to be positive (see \eqref{PredVal}). This
corresponds to the pattern observed in the teacher data in Figure
\ref{QTI}. At a strength of 0.4 and up we see that the scores on the
circle range on average between 0\(^\circ\) and 100\(^\circ\).

\subsubsection{Model fit}

Table \ref{tab:ModelFit} shows the values fof the PLSL criterion for the
linear and circular outcomes of the four cylindrical models fit to the
teacher data.\newline \indent The CL-PN and CL-GPN models have the best
out-of-sample predictive performance for the linear outcome. They show
roughly the same performance because they model the linear outcome in
the same way. We should note that even though the predictive performance
of the Abe-Ley model for the linear outcome is worst on average, the
standard deviation of the cross-validation estimates is rather large.
This means that in some samples, the Abe-Ley model shows a lower PLSL
value than the average of 25.49\newline \indent The Abe-Ley model has
the best out-of-sample predictive performance for the circular outcome.
This would suggest that for the circular variable a slightly skewed
distribution fits best. However, both the GPN-SSN and the CL-GPN models
fit much worse even though the distribution for the circular outcome in
these models can also take a skewed shape. It should be noted that the
standard deviation of the cross-validation estimates is rather large for
the Abe-Ley and the CL-GPN model. It is possible that these large
standard deviations for the PLSL are caused by the fact that they are
computed for a relatively small sample size, but this does not explain
why the PLSL has a large standard deviation for only a few cylindrical
models and not for all.\newline \indent In this situation, where one
model fits the linear outcome best and another one fits the circular
outcome best, it is hard to determine which model we should choose. In
this case the results for the CL-PN /CL-GPN and Abe-Ley model are quite
different regarding the effect of self-efficacy on the linear outcome
(strength of interpersonal behavior). Because the Abe-Ley fit for the
linear part is worst we would choose to trust the results for the CL-PN
and CL-GPN model here. For the circular part however the results of the
CL-PN/CL-GPN model do not differ as much from the Abe-Ley model and we
reach the same conclusion for both models, namely that the effect of
self-efficacy on type of interpersonal behavior is not very strong.
Therefore we would prefer the CL-PN/CL-GPN models in this case because
where it matters in terms of interpretation (the linear part) they show
better fit.

\begin{table}

\caption{\label{tab:ModelFit}PLSL criteria, cross-validation mean and standard deviation, for the circular and linear outcome in the four cylindrical models}
\centering
\begin{tabular}[t]{lrlrl}
\toprule
\multicolumn{1}{c}{Model} & \multicolumn{2}{c}{Circular} & \multicolumn{2}{c}{Linear} \\
\cmidrule(l{2pt}r{2pt}){1-1} \cmidrule(l{2pt}r{2pt}){2-3} \cmidrule(l{2pt}r{2pt}){4-5}
  & mean & sd & mean & sd\\
\midrule
CL-PN & 82.96 & (9.47) & -17.65 & (3.70)\\
CL-GPN & 78.21 & (14.53) & -18.30 & (3.00)\\
Abe-Ley & 31.97 & (22.07) & 25.49 & (17.46)\\
GPN-SSN & 107.10 & (10.52) & -2.37 & (7.01)\\
\bottomrule
\end{tabular}
\end{table}

\section{Discussion}\label{Discussion}

In this paper we modified four models for cylindrical data in such a way
that they include a regression of both the linear and circular outcome
onto a set of covariates. Subsequently we have shown how these four
methods can be used to analyze a dataset on the interpersonal behavior
of teachers. In this final section we will first comment on what
researchers can gain by using cylindrical models for the teacher data.
Subsequently we will comment on the differences between the cylindrical
models that were introduced in this paper.\newline \indent Concerning
the teacher data, the advantage of cylindrical data analysis is that we
were able to analyze the information about the type and strength of
interpersonal behavior simultaneously. In previous research, the two
components of the interpersonal circumplex (\emph{i.e.}, Agency and
Communion) were analyzed separately. Such an approach also provides
information about the strength of teachers' score on Agency and
Communion, yet a large portion of information about the combination of
Agency and Communion, which describes the type of behavior that is
observed, gets lost. A first solution to include both dimensions as a
circular variable in data analysis was described by Cremers et al.
(2018a). A downside of that analysis was that information about the
strength of the specific type of interpersonal behavior could not be
retained. In the present study, we have shown how using cylindrical
models can simultaneously model the information about the type of and
strength of interpersonal behavior and how these are influenced by
teachers' self-efficacy in classroom management. Although we do not find
any strong effects of self-efficacy on either the type or strength of
behavior, the four cylindrical models do provide a way of analyzing and
interpreting this effect. This is beneficial for future research in
which we may want to investigate the effect of further covariates on
data from the circumplex. \newline \indent Furthermore, in addition to
being able to assess the influence of covariates, the cylindrical models
also provide information about the dependence between the type and
strength of interpersonal behavior. We found that stronger behavior is
associated with higher scores on the Communion and in some models also
the Agency dimension. This implies that teachers whose type of
interpersonal behavior ranges between 0\(^\circ\) and 90\(^\circ\), the
\enquote{helpful} and \enquote{directing} subtypes are stronger in their
behavior than teachers of the other subtypes.\newline \indent As
mentioned in the introduction, data from the interpersonal circumplex is
not the only type of cylindrical data that occurs in psychology. The
methods presented in this paper are also of use for research on human
navigation and eye-tracking research. Furthermore, even though
cylindrical models are already used in fields outside of psychology, the
addition of a regression structure to the models is of use in these
fields as well. \newline \indent In terms of interpretability, the CL-PN
and Abe-Ley models perform best out of the four cylindrical models. In
the CL-GPN and GPN-SSN models the interpretation of the parameters of
the circular outcome component is not straightforward, if at all
possible. This is caused by the fact that in addition to the mean vector
the covariance matrix of the GPN distribution affects the location of
the circular data, making it difficult to compute regression
coefficients on the circle. Wang and Gelfand (2013) state that Monte
Carlo integration can be used to compute a circular mean and variance
for the GPN distribution. In future research, this solution might be
applied to the methods of Cremers et al. (2018b) in order to compute
circular coefficients for GPN models.\newline \indent In terms of
flexibility the GPN-SSN model scores best. Multiple linear and circular
outcomes can be included and we can thus apply the model to multivariate
cylindrical data. In addition the GPN-SSN, the CL-GPN and CL-PN models
are extendable to a mixed-effects structure and can thus also be fit to
longitudinal data (see Nuñez-Antonio and Gutiérrez-Peña (2014) and
Hernandez-Stumpfhauser et al. (2016) for hierarchical/mixed-effects
models for the PN and GPN distributions respectively). For the Abe-Ley
model this may also be possible but has not been done in previous
research for the conditional distribution of its circular outcome
(sine-skewed von Mises). Concerning asymmetry, both the GPN-SSN as well
as the Abe-Ley model allow for non-symmetrical shapes of the
distributions of both the linear and circular outcome, while the CL-GPN
model permits an asymmetric circular outcome.\newline \indent The four
cylindrical models that were modified to the regression context in this
paper are not the only cylindrical distributions available from the
literature. Other interesting cylindrical distributions have been
introduced by Fernández-Durán (2007), Kato and Shimizu (2008) and
Sugasawa (2015) (for more references we refer to Chapter 2 of Ley and
Verdebout (2017)). In the present study we have decided not to include
these distributions for reasons of space, complexity of the models and
ease of implementing a regression structure. In future research however
it would be interesting to investigate other types of cylindrical
distributions as well in order to compare the interpretability,
flexibility and model fit to the models developed in the present
study.\newline

\section*{References}

\hypertarget{refs}{}
\leavevmode\hypertarget{ref-abe2017tractable}{}%
Abe, T., \& Ley, C. (2017). A tractable, parsimonious and flexible model
for cylindrical data, with applications. \emph{Econometrics and
Statistics}, \emph{4}, 91--104.
doi:\href{https://doi.org/10.1016/j.ecosta.2016.04.001}{10.1016/j.ecosta.2016.04.001}

\leavevmode\hypertarget{ref-chrastil2017rotational}{}%
Chrastil, E. R., \& Warren, W. H. (2017). Rotational error in path
integration: Encoding and execution errors in angle reproduction.
\emph{Experimental Brain Research}, \emph{235}(6), 1885--1897.
doi:\href{https://doi.org/10.1007/s00221-017-4910-y}{10.1007/s00221-017-4910-y}

\leavevmode\hypertarget{ref-Claessens2016side}{}%
Claessens, L. C. (2016). \emph{Be on my side i'll be on your side :
Teachers' perceptions of teacher--student relationships} (PhD thesis).

\leavevmode\hypertarget{ref-Cremers2018Assessing}{}%
Cremers, J., Mainhard, M. T., \& Klugkist, I. (2018a). Assessing a
Bayesian embedding approach to circular regression models.
\emph{Methodology}, \emph{14}(2), 69--81.

\leavevmode\hypertarget{ref-CremersMulderKlugkist2017}{}%
Cremers, J., Mulder, K. T., \& Klugkist, I. (2018b). Circular
interpretation of regression coefficients. \emph{British Journal of
Mathematical and Statistical Psychology}, \emph{71}(1), 75--95.
doi:\href{https://doi.org/10.1111/bmsp.12108}{10.1111/bmsp.12108}

\leavevmode\hypertarget{ref-fernandez2007models}{}%
Fernández-Durán, J. (2007). Models for circular--linear and
circular--circular data constructed from circular distributions based on
nonnegative trigonometric sums. \emph{Biometrics}, \emph{63}(2),
579--585.
doi:\href{https://doi.org/10.1111/j.1541-0420.2006.00716.x}{10.1111/j.1541-0420.2006.00716.x}

\leavevmode\hypertarget{ref-fisher1995statistical}{}%
Fisher, N. I. (1995). \emph{Statistical analysis of circular data}.
Cambridge: Cambridge University Press.

\leavevmode\hypertarget{ref-garcia2014test}{}%
García-Portugués, E., Barros, A. M., Crujeiras, R. M.,
González-Manteiga, W., \& Pereira, J. (2014). A test for
directional-linear independence, with applications to wildfire
orientation and size. \emph{Stochastic Environmental Research and Risk
Assessment}, \emph{28}(5), 1261--1275.
doi:\href{https://doi.org/10.1007/s00477-013-0819-6}{10.1007/s00477-013-0819-6}

\leavevmode\hypertarget{ref-garcia2013exploring}{}%
García-Portugués, E., Crujeiras, R. M., \& González-Manteiga, W. (2013).
Exploring wind direction and SO2 concentration by circular--linear
density estimation. \emph{Stochastic Environmental Research and Risk
Assessment}, \emph{27}(5), 1055--1067.
doi:\href{https://doi.org/10.1007/s00477-012-0642-5}{10.1007/s00477-012-0642-5}

\leavevmode\hypertarget{ref-BDA}{}%
Gelman, A., Carlin, J., Stern, H., Dunson, D., Vehtari, A., \& Rubin, D.
(2014). \emph{Bayesian data analysis} (3rd ed.). Boca Raton, FL: Chapman
\& Hall/CRC.

\leavevmode\hypertarget{ref-gneiting2007strictly}{}%
Gneiting, T., \& Raftery, A. E. (2007). Strictly proper scoring rules,
prediction, and estimation. \emph{Journal of the American Statistical
Association}, \emph{102}(477), 359--378.
doi:\href{https://doi.org/10.1198/016214506000001437}{10.1198/016214506000001437}

\leavevmode\hypertarget{ref-gurtman2009exploring}{}%
Gurtman, M. B. (2009). Exploring personality with the interpersonal
circumplex. \emph{Social and Personality Psychology Compass},
\emph{3}(4), 601--619.
doi:\href{https://doi.org/10.1111/j.1751-9004.2009.00172.x}{10.1111/j.1751-9004.2009.00172.x}

\leavevmode\hypertarget{ref-gurtman2011reasoning}{}%
Gurtman, M. B. (2011). Handbook of interpersonal psychology. In L. M.
Horowitz \& S. Strack (Eds.), (pp. 299--311). New York: Wiley.

\leavevmode\hypertarget{ref-hernandez2016general}{}%
Hernandez-Stumpfhauser, D., Breidt, F. J., \& Woerd, M. J. van der.
(2016). The general projected normal distribution of arbitrary
dimension: Modeling and Bayesian inference. \emph{Bayesian Analysis},
\emph{12}(1), 113--133.
doi:\href{https://doi.org/10.1214/15-BA989}{10.1214/15-BA989}

\leavevmode\hypertarget{ref-horowitz2010handbook}{}%
Horowitz, L. M., \& Strack, S. (2011). \emph{Handbook of interpersonal
psychology: Theory, research, assessment, and therapeutic
interventions}. Hoboken, NJ: John Wiley \& Sons.

\leavevmode\hypertarget{ref-jammalamadaka2001topics}{}%
Jammalamadaka, S. R., \& Sengupta, A. (2001). \emph{Topics in circular
statistics} (Vol. 5). World Scientific.

\leavevmode\hypertarget{ref-johnson1978some}{}%
Johnson, R. A., \& Wehrly, T. E. (1978). Some angular-linear
distributions and related regression models. \emph{Journal of the
American Statistical Association}, \emph{73}(363), 602--606.

\leavevmode\hypertarget{ref-kato2008dependent}{}%
Kato, S., \& Shimizu, K. (2008). Dependent models for observations which
include angular ones. \emph{Journal of Statistical Planning and
Inference}, \emph{138}(11), 3538--3549.
doi:\href{https://doi.org/10.1016/j.jspi.2006.12.009}{10.1016/j.jspi.2006.12.009}

\leavevmode\hypertarget{ref-lagona2015hidden}{}%
Lagona, F., Picone, M., Maruotti, A., \& Cosoli, S. (2015). A hidden
Markov approach to the analysis of space--time environmental data with
linear and circular components. \emph{Stochastic Environmental Research
and Risk Assessment}, \emph{29}(2), 397--409.
doi:\href{https://doi.org/10.1007/s00477-014-0919-y}{10.1007/s00477-014-0919-y}

\leavevmode\hypertarget{ref-leary1957}{}%
Leary, T. (1957). \emph{An interpersonal diagnosis of personality}. New
York: Ronald Press Company.

\leavevmode\hypertarget{ref-ley2017modern}{}%
Ley, C., \& Verdebout, T. (2017). \emph{Modern directional statistics}.
Chapman \& Hall/CRC Press. Boca Raton, FL.

\leavevmode\hypertarget{ref-locke2007selfefficacy}{}%
Locke, K. D., \& Sadler, P. (2007). Self-efficacy, values, and
complementarity in dyadic interactions: Integrating interpersonal and
social-cognitive theory. \emph{Personality and Social Psychology
Bulletin}, \emph{33}, 94--109.
doi:\href{https://doi.org/10.1177/0146167206293375}{10.1177/0146167206293375}

\leavevmode\hypertarget{ref-mardia2000directional}{}%
Mardia, K. V., \& Jupp, P. E. (2000). \emph{Directional statistics}
(Vol. 494). Chichester, England: Wiley.

\leavevmode\hypertarget{ref-mardia1978model}{}%
Mardia, K. V., \& Sutton, T. W. (1978). A model for cylindrical
variables with applications. \emph{Journal of the Royal Statistical
Society Series B (Methodological)}, \emph{40}(2), 229--233.

\leavevmode\hypertarget{ref-mastrantonio2018joint}{}%
Mastrantonio, G. (2018). The joint projected normal and skew-normal: A
distribution for poly-cylindrical data. \emph{Journal of Multivariate
Analysis}, \emph{165}, 14--26.
doi:\href{https://doi.org/10.1016/j.jmva.2017.11.006}{10.1016/j.jmva.2017.11.006}

\leavevmode\hypertarget{ref-mastrantonio2015bayesian}{}%
Mastrantonio, G., Maruotti, A., \& Jona-Lasinio, G. (2015). Bayesian
hidden Markov modelling using circular-linear general projected normal
distribution. \emph{Environmetrics}, \emph{26}(2), 145--158.
doi:\href{https://doi.org/10.1002/env.2326}{10.1002/env.2326}

\leavevmode\hypertarget{ref-nunez2014bayesian}{}%
Nuñez-Antonio, G., \& Gutiérrez-Peña, E. (2014). A Bayesian model for
longitudinal circular data based on the projected normal distribution.
\emph{Computational Statistics \& Data Analysis}, \emph{71}, 506--519.
doi:\href{https://doi.org/10.1016/j.csda.2012.07.025}{10.1016/j.csda.2012.07.025}

\leavevmode\hypertarget{ref-nunez2011bayesian}{}%
Nuñez-Antonio, G., Gutiérrez-Peña, E., \& Escarela, G. (2011). A
Bayesian regression model for circular data based on the projected
normal distribution. \emph{Statistical Modelling}, \emph{11}(3),
185--201.
doi:\href{https://doi.org/10.1177/1471082X1001100301}{10.1177/1471082X1001100301}

\leavevmode\hypertarget{ref-pennings2017phd}{}%
Pennings, H. J. M. (2017a). \emph{Interpersonal dynamics in
teacher-student interactions and relationships} (PhD thesis). Enschede:
Ipskamp drukkers.

\leavevmode\hypertarget{ref-pennings2017complexity}{}%
Pennings, H. J. M. (2017b). Using a complexity approach to study the
interpersonal dynamics in teacher-student interactions: A case study of
two teachers. \emph{Complicity: An International Journal of Complexity
and Education}, \emph{14}(2), 88--103.

\leavevmode\hypertarget{ref-pennings2018interpersonal}{}%
Pennings, H. J. M., Brekelmans, M., Sadler, P., Claessens, L. C.,
\VANDER{Want}{Van der}{van der} Want, A. C., \& Tartwijk, J. van.
(2018). Interpersonal adaptation in teacher-student interaction.
\emph{Learning and Instruction}, \emph{55}, 41--57.
doi:\href{https://doi.org/10.1016/j.learninstruc.2017.09.005}{10.1016/j.learninstruc.2017.09.005}

\leavevmode\hypertarget{ref-plutchik1997general}{}%
Plutchik, R. (1997). Circumplex models of personality and emotions. In
C. Plutchik Robert. (Ed.), (pp. 17--45). Washington, DC, US: American
Psychological Association.

\leavevmode\hypertarget{ref-rayner200935th}{}%
Rayner, K. (2009). The 35th Sir Frederick Bartlett lecture: Eye
movements and attention in reading, scene perception, and visual search.
\emph{Quarterly Journal of Experimental Psychology}, \emph{62}(8),
1457--1506.
doi:\href{https://doi.org/10.1080/17470210902816461}{10.1080/17470210902816461}

\leavevmode\hypertarget{ref-russell1980circumplex}{}%
Russell, J. A. (1980). A circumplex model of affect. \emph{Journal of
Personality and Social Psychology}, \emph{39}(6), 1161--1178.
doi:\href{https://doi.org/10.1037/h0077714}{10.1037/h0077714}

\leavevmode\hypertarget{ref-sahu2003new}{}%
Sahu, S. K., Dey, D. K., \& Branco, M. D. (2003). A new class of
multivariate skew distributions with applications to Bayesian regression
models. \emph{Canadian Journal of Statistics}, \emph{31}(2), 129--150.
doi:\href{https://doi.org/10.2307/3316064}{10.2307/3316064}

\leavevmode\hypertarget{ref-schwartz1992values}{}%
Schwartz, S. H. (1992). Advances in experimental social psychology. In
M. Zanna (Ed.), (Vol. 25, pp. 1--65). San Diego, CA: Academic Press.

\leavevmode\hypertarget{ref-sugasawa2015flexible}{}%
Sugasawa, S., S. (2015). \emph{A flexible family of distributions on the
cylinder}. Retrieved from
\href{arXiv:\%201501.06332v2}{arXiv: 1501.06332v2}

\leavevmode\hypertarget{ref-wang2012directional}{}%
Wang, F., \& Gelfand, A. E. (2013). Directional data analysis under the
general projected normal distribution. \emph{Statistical Methodology},
\emph{10}(1), 113--127.
doi:\href{https://doi.org/10.1016/j.stamet.2012.07.005}{10.1016/j.stamet.2012.07.005}

\leavevmode\hypertarget{ref-vanderWant2015role}{}%
\VANDER{Want}{Van der}{van der} Want, A. C. (2015). \emph{Teachers'
interpersonal role identity.} (PhD thesis).

\leavevmode\hypertarget{ref-want2018selfefficacy}{}%
\VANDER{Want}{Van der}{van der} Want, A. C., Den Brok, P., Beijaard, D.,
Brekelmans, M., Claessens, L. C., \& Pennings, H. J. M. (2018). The
relation between teachers' interpersonal role identity and their
self-efficacy, burnout and work engagement. \emph{Professional
Development in Education}, Advance Online Publication.
doi:\href{https://doi.org/10.1080/19415257.2018.1511453}{10.1080/19415257.2018.1511453}

\leavevmode\hypertarget{ref-wiggins1996history}{}%
Wiggins, J. (1996). An informal history of the interpersonal circumplex
tradition. \emph{Journal of Personality Assessment}, \emph{66}(2),
217--233.

\leavevmode\hypertarget{ref-wright2009integrating}{}%
Wright, A. G., Pincus, A. L., Conroy, D. E., \& Hilsenroth, M. J.
(2009). Integrating methods to optimize circumplex description and
comparison of groups. \emph{Journal of Personality Assessment},
\emph{91}(4), 311--322.
doi:\href{https://doi.org/10.1080/00223890902935696}{10.1080/00223890902935696}

\leavevmode\hypertarget{ref-wubbels2006interpersonal}{}%
Wubbels, T., Brekelmans, M., Brok, P. den, \& Tartwijk, J. van. (2006).
Handbook of classroom management: Research practice and contemporary
issues. In C. Evertson \& C. S. Weinstein (Eds.), \emph{Handbook of
classroom management: Research, practice, and contemporary issues} (pp.
1161--1191). Malwah, NJ: Lawrence Erlbaum Associates.


\end{document}
