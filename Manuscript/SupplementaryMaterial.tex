\documentclass[12pt,]{article}
\usepackage{lmodern}
\usepackage{amssymb,amsmath}
\usepackage{ifxetex,ifluatex}
\usepackage{fixltx2e} % provides \textsubscript
\ifnum 0\ifxetex 1\fi\ifluatex 1\fi=0 % if pdftex
  \usepackage[T1]{fontenc}
  \usepackage[utf8]{inputenc}
\else % if luatex or xelatex
  \ifxetex
    \usepackage{mathspec}
  \else
    \usepackage{fontspec}
  \fi
  \defaultfontfeatures{Ligatures=TeX,Scale=MatchLowercase}
\fi
% use upquote if available, for straight quotes in verbatim environments
\IfFileExists{upquote.sty}{\usepackage{upquote}}{}
% use microtype if available
\IfFileExists{microtype.sty}{%
\usepackage{microtype}
\UseMicrotypeSet[protrusion]{basicmath} % disable protrusion for tt fonts
}{}
\usepackage[margin=1in]{geometry}
\usepackage{hyperref}
\hypersetup{unicode=true,
            pdfborder={0 0 0},
            breaklinks=true}
\urlstyle{same}  % don't use monospace font for urls
\usepackage{graphicx,grffile}
\makeatletter
\def\maxwidth{\ifdim\Gin@nat@width>\linewidth\linewidth\else\Gin@nat@width\fi}
\def\maxheight{\ifdim\Gin@nat@height>\textheight\textheight\else\Gin@nat@height\fi}
\makeatother
% Scale images if necessary, so that they will not overflow the page
% margins by default, and it is still possible to overwrite the defaults
% using explicit options in \includegraphics[width, height, ...]{}
\setkeys{Gin}{width=\maxwidth,height=\maxheight,keepaspectratio}
\IfFileExists{parskip.sty}{%
\usepackage{parskip}
}{% else
\setlength{\parindent}{0pt}
\setlength{\parskip}{6pt plus 2pt minus 1pt}
}
\setlength{\emergencystretch}{3em}  % prevent overfull lines
\providecommand{\tightlist}{%
  \setlength{\itemsep}{0pt}\setlength{\parskip}{0pt}}
\setcounter{secnumdepth}{0}
% Redefines (sub)paragraphs to behave more like sections
\ifx\paragraph\undefined\else
\let\oldparagraph\paragraph
\renewcommand{\paragraph}[1]{\oldparagraph{#1}\mbox{}}
\fi
\ifx\subparagraph\undefined\else
\let\oldsubparagraph\subparagraph
\renewcommand{\subparagraph}[1]{\oldsubparagraph{#1}\mbox{}}
\fi

%%% Use protect on footnotes to avoid problems with footnotes in titles
\let\rmarkdownfootnote\footnote%
\def\footnote{\protect\rmarkdownfootnote}

%%% Change title format to be more compact
\usepackage{titling}

% Create subtitle command for use in maketitle
\newcommand{\subtitle}[1]{
  \posttitle{
    \begin{center}\large#1\end{center}
    }
}

\setlength{\droptitle}{-2em}

  \title{}
    \pretitle{\vspace{\droptitle}}
  \posttitle{}
    \author{}
    \preauthor{}\postauthor{}
    \date{}
    \predate{}\postdate{}
  
\usepackage{multirow}
\usepackage{appendix}
\usepackage{color}
\usepackage{hyperref}
\usepackage{subcaption}
\setlength\parindent{24pt}
\usepackage{setspace}

\doublespacing
\DeclareRobustCommand{\VANDER}[3]{#2}
\DeclareRobustCommand{\VAN}[3]{#2}
\DeclareRobustCommand{\DEN}[3]{#2}

\begin{document}

\section{Supplementary Material}\label{Appendix}

In this Supplementary Material we give a more detailed description of
the cylindrical regression models and outline the MCMC procedures to fit
them. R-code for the MCMC sampler and the analysis of the teacher data
can be found here:
\url{https://github.com/joliencremers/CylindricalComparisonCircumplex}.
Note that the dimensions of the objects (design matrices, mean vectors,
etc.) are those that were used in the analysis of the teacher data where
we have 1 circular component, 1 linear component and estimate an
intercept and regression coefficient for the covariate self-efficacy.
Note that for the regression of the linear component in the CL-PN and
CL-GPN models we also have the sine and cosine of the circular component
in the regression equation, this makes the vector with regression
coefficients, \(\boldsymbol{\gamma}\), four-dimensional.

\section{Four cylindrical regression models}\label{Models}

\subsection{The modified CL-PN and modified CL-GPN  models}\label{CL-(G)PN}

Following Mastrantonio, Maruotti, \& Jona-Lasinio (2015) we consider in
this section two models where the prediction equation for the linear
component is specified as \begin{equation}\label{linpredCLPNCLGPN}
\hat{y_i} = \gamma_0 + \gamma_{cos}*\cos(\theta_i)*r_i + \gamma_{sin}*\sin(\theta_i)*r_i + \gamma_1*x_1 + \dots + \gamma_q*x_q,
\end{equation} \noindent where \(r_i\) is a realization of the
unobserved the random variable \(R\geq0\) that will be introduced below,
\(\gamma_0, \gamma_{cos}, \gamma_{sin}, \gamma_1, \dots, \gamma_q\) are
the intercept and regression coefficients and \(x_1, \dots, x_q\) are
the \(q\) covariates. In both of these models the conditional
distribution of \(Y\) given \(\Theta=\theta\) and \(R = r\) is given by
\begin{equation}\label{ycondtheta}
f(y \mid \theta, r) = \frac{1}{\sqrt{2\pi\sigma^2}}\exp\left[-\frac{(y - (\gamma_0 + \gamma_1x_1 + \dots + \gamma_qx_q+c))^{2}}{2\sigma^2}\right],\nonumber
\end{equation} \noindent where
\(c = \begin{bmatrix} r \cos(\theta) \\ r\sin(\theta) \end{bmatrix}^t \begin{bmatrix} \gamma_{cos} \\ \gamma_{sin} \end{bmatrix}\),
\(r \geq 0\). The linear component thus has a normal distribution
conditional on \(\Theta\) and \(R\) and contains already linear
covariates \(x_1, \dots, x_q\) in its location part.\newline \indent For
the circular component we assume either a projected normal (PN) or a
general projected normal (GPN) distribution. These distributions arise
from the radial projection of a distribution defined on the plane onto
the circle. The relation between a bivariate vector \(\boldsymbol{S}\)
in the plane and the circular component \(\Theta\) is defined as follows
\begin{equation}\label{projection}
\boldsymbol{S} = \begin{bmatrix} S^{I} \\ S^{II} \end{bmatrix} = R\boldsymbol{u} = \begin{bmatrix} R \cos (\Theta) \\  R\sin (\Theta) \end{bmatrix},
\end{equation} \noindent where \(R = \mid\mid \boldsymbol{S} \mid\mid\),
the Euclidean norm of the bivariate vector \(\boldsymbol{S}\). In the PN
distribution we assume
\(\boldsymbol{S} \sim N_2(\boldsymbol{\mu}, \boldsymbol{I})\) and in the
GPN we assume
\(\boldsymbol{S} \sim N_2(\boldsymbol{\mu}, \boldsymbol{\Sigma})\) where
\(\boldsymbol{\mu} \in \mathbb{R}^2\),
\(\boldsymbol{\Sigma} = \begin{bmatrix} \tau^2 + \xi^2 & \xi\\ \xi & 1 \end{bmatrix}\),
and \(\xi,\tau \in (-\infty, +\infty)\) (as in Hernandez-Stumpfhauser,
Breidt, \& \VANDER{Woerd}{Van der}{van der} Woerd (2016)). This leads to
the circular-linear PN (CL-PN) and circular-linear GPN (CL-GPN)
distributions. We will now detail how we modify both cylindrical
distributions to also incorporate covariates for the circular part.

\subsubsection{The modified CL-PN distribution}

Following Nuñez-Antonio, Gutiérrez-Peña, \& Escarela (2011), the joint
density of \(\Theta\) and \(R\) for the PN distribution equals
\begin{equation}\label{pnreg}
f(\theta,r \mid \boldsymbol{\mu}, \boldsymbol{I}) = \frac{1}{2\pi} \exp\{-0.5 \mid\mid\boldsymbol{\mu}^2\mid\mid\}\exp\{-0.5\left[r^2 -2r(\boldsymbol{u}^t\boldsymbol{\mu}) \right]\},
\end{equation} \noindent where
\(\boldsymbol{u}= \begin{bmatrix} \cos (\theta) \\ \sin (\theta) \end{bmatrix}\)
and \(r\) is defined in \eqref{projection}. In a regression setup the
outcome components \(\theta_i,r_i\) for each individual
\(i = 1, \dots, n\), where \(n\) is the sample size, are generated
independently from the distribution with density \eqref{pnreg}. The mean
vector \(\boldsymbol{\mu}_i \in \mathbb{R}^2\) is then defined as
\(\boldsymbol{\mu}_i = \boldsymbol{B}^t\boldsymbol{z}_i\) where the
vector \(\boldsymbol{z}_i\) is a vector of dimension \(p + 1\) that
contains the covariate values and the value 1 to estimate an intercept
and
\(\boldsymbol{B} = (\boldsymbol{\beta}^{I}, \boldsymbol{\beta}^{II})\)
contains the regression coefficients and intercepts.

\subsubsection{The modified CL-GPN distribution}

Following Wang \& Gelfand (2013) and Hernandez-Stumpfhauser et al.
(2016) the joint density of \(R\) and \(\Theta\) for the GPN
distribution equals \begin{equation}\label{gpnreg}
f(\theta, r \mid \boldsymbol{\mu}, \boldsymbol{\Sigma}) = \frac{r}{2\pi\tau} \exp\left[ -\frac{(r\boldsymbol{u}-\boldsymbol{\mu})^{t}\boldsymbol{\Sigma}^{-1}(r\boldsymbol{u}-\boldsymbol{\mu})}{2\tau^2}\right],
\end{equation} \noindent where we recall that
\(\boldsymbol{\Sigma} = \begin{bmatrix} \tau^2 + \xi^2 & \xi\\ \xi & 1 \end{bmatrix}\).
In a regression setup the outcome components \(\theta_i\) and \(r_i\)
for each individual are generated independently from \eqref{gpnreg}. The
mean vector \(\boldsymbol{\mu}_i \in \mathbb{R}^2\) is defined in the
same way via covariates as for the modified CL-PN distribution.

\subsubsection{Parameter estimation}

Both cylindrical models introduced here are estimated using Markov Chain
Monte Carlo (MCMC) methods based on Nuñez-Antonio et al. (2011), Wang \&
Gelfand (2013) and Hernandez-Stumpfhauser et al. (2016) for the
regression of the circular component.

\subsection{The modified Abe-Ley model}\label{WeiSSVM}

This model is an extension of the cylindrical model introduced in Abe \&
Ley (2017) to the regression context. The joint density of \(\Theta\)
and \(Y\), in this model defined only on the positive real half-line
\([0, + \infty)\), reads \begin{equation}\label{WeiSSVMdensity}
f(\theta, y) = \frac{\alpha\nu^\alpha}{2\pi\cosh(\kappa)}
                 (1 +\lambda\sin(\theta - \mu_c))
                 y^{\alpha-1}
                 \exp[-(\nu y)^{\alpha}(1-\tanh(\kappa)\cos(\theta - \mu_c))],
\end{equation} \noindent where \(\alpha > 0\) is a linear shape
parameter, \(\kappa > 0\) and \(\lambda \in [-1, 1]\) are circular
concentration and skewness parameters with \(\kappa\) also regulating
the circular-linear dependence. Our modification occurs at the level of
the linear scale parameter \(\nu>0\) and circular location parameter
\(\mu_c\in [0, 2\pi)\), both of which we express in terms of covariates:
\(\nu_i = \exp(\boldsymbol{x}_i^t\boldsymbol{\gamma}) > 0\) and
\(\mu_{c,i} = \beta_0 + 2\tan^{-1}(\boldsymbol{z}_i^t\boldsymbol{\beta})\).
The parameter \(\boldsymbol{\gamma}\) is a vector of \(q\) regression
coefficients \(\gamma_j \in (-\infty, +\infty)\) for the prediction of
\(y\) where \(j = 0, \dots, q\) and \(\nu_0\) is the intercept. The
parameter \(\beta_0 \in [0, 2\pi)\) is the intercept and
\(\boldsymbol{\beta}\) is a vector of \(p\) regression coefficients
\(\beta_j \in (-\infty, +\infty)\) for the prediction of \(\theta\)
where \(j = 1, \dots, p\). The vector \(\boldsymbol{x}_i\) is a vector
of predictor values for the prediction of \(y\) and \(\boldsymbol{z}_i\)
is a vector of predictor values for the prediction of \(\theta\). In a
regression setup the outcome component vector \((\theta_i, y_i)^t\) for
each individual is generated independently from the modified density
\eqref{WeiSSVMdensity}.\newline \indent As in Abe \& Ley (2017), the
conditional distribution of \(Y\) given \(\Theta=\theta\) is a Weibull
distribution with shape \(\alpha\) and scale
\(\nu(1-\tanh(\kappa)\cos(\theta - \mu_c))^{1/\alpha}\) and the
conditional distribution of \(\Theta\) given \(Y=y\) is a sine skewed
von Mises distribution with location parameter \(\mu_c\) and
concentration parameter \((\nu y)^\alpha\tanh(\kappa)\). The
log-likelihood for this model equals
\begin{align}\label{WeiSSVMLikelihood}
l(\alpha, \boldsymbol{\gamma}, \lambda, \kappa, \boldsymbol{\beta}) 
   &= n[\ln(\alpha) - \ln(2\pi\cosh(\kappa))] + \alpha \sum^{n}_{i = 1} \boldsymbol{x}_i^t\boldsymbol{\gamma} \nonumber\\
   &\:\:\:\:+\sum^{n}_{i = 1} \ln(1 +\lambda\sin(\theta_i - (\beta_0 + 2\tan^{-1}(\boldsymbol{z}_i^t\boldsymbol{\beta})))) 
   +(\alpha-1)\sum^{n}_{i = 1} \ln(y_i) \nonumber\\
   &\:\:\:\:-\sum^{n}_{i = 1}( \exp(\boldsymbol{x}_i^t\boldsymbol{\gamma})y_i)^{\alpha}(1-\tanh(\kappa)\cos(\theta_i - (\beta_0 + 2\tan^{-1}(\boldsymbol{z}_i^t\boldsymbol{\beta})))).\nonumber
\end{align} \noindent We can use numerical optimization (Nelder-Mead) to
find solutions for the maximum likelihood (ML) estimates for the
parameters of the model.

\subsection{Modified joint projected and skew normal (GPN-SSN)}\label{CL-GPN_multivariate}

This model is an extension of the cylindrical model introduced by
Mastrantonio (2018) to the regression context. Both models contain \(m\)
independent circular components and \(w\) independent linear components.
The circular components
\(\boldsymbol{\Theta} = (\boldsymbol{\Theta}_1, \dots,  \boldsymbol{\Theta}_m)\)
are modelled together by a multivariate GPN distribution. The joint
distribution of \(\boldsymbol{\Theta}\) and \(\boldsymbol{R}\) can thus
be modeled as the product of (\ref{gpnreg}) for each of the \(m\)
circular components. The linear components
\(\boldsymbol{Y} = (\boldsymbol{Y}_1,  \dots, \boldsymbol{Y}_w)\) are
modelled together by a multivariate skew normal distribution (Sahu, Dey,
\& Branco, 2003). Because the GPN distribution is modelled using a
so-called augmented representation (as in \eqref{projection} and
\eqref{gpnreg}) it is convenient to use a similar tactic for modelling
the multivariate skew normal distribution. Following Mastrantonio (2018)
the linear components are represented as
\[\boldsymbol{Y} = \boldsymbol{M}_y + \boldsymbol{\Lambda}\boldsymbol{D} + \boldsymbol{H},\]
\noindent where \(\boldsymbol{M}_y\) is a mean vector for the linear
component \(\boldsymbol{Y}\),
\(\boldsymbol{\Lambda} = \text{diag}(\boldsymbol{\lambda})\) is a
\(w \times w\) diagonal matrix with diagonal elements
\(\lambda_1, \dots, \lambda_w\) (skewness parameters),
\(\boldsymbol{D} \sim HN_w(\boldsymbol{0}_w, \boldsymbol{I}_w)\), a
\(w\)-dimensional half normal distribution (Olmos, Varela, Gómez, \&
Bolfarine, 2012), and
\(\boldsymbol{H} \sim N_w(\boldsymbol{0}_w, \boldsymbol{\Sigma}_y)\).
This means that, conditional on the auxiliary data \(\boldsymbol{D}\),
\(\boldsymbol{Y}\) is normally distributed with mean
\(\boldsymbol{M}_y + \boldsymbol{\Lambda}\boldsymbol{D}\) and covariance
matrix \(\boldsymbol{\Sigma}_y\). The joint density for
\((\boldsymbol{Y}^t, \boldsymbol{D}^t)^t\) is defined as:
\begin{equation}\label{YDjoint}
f(\boldsymbol{y}, \boldsymbol{d}) = 2^w\phi_w(\boldsymbol{y} \mid \boldsymbol{M}_y + \boldsymbol{\Lambda}\boldsymbol{d}, \boldsymbol{\Sigma}_y) \phi_w(\boldsymbol{d} \mid \boldsymbol{0}_w, \boldsymbol{I}_w),\nonumber
\end{equation} \noindent where
\(\phi_\ell(\cdot|\boldsymbol{M}_\ell,\boldsymbol{\Sigma}_\ell)\) stands
for the \(\ell\)-dimensional normal density with mean vector
\(\boldsymbol{M}_\ell\) and covariance \(\boldsymbol{\Sigma}_\ell\). As
in Mastrantonio (2018) dependence between the linear and circular
component is created by modelling the augmented representations of
\(\boldsymbol{\Theta}\) and \(\boldsymbol{Y}\) together in a \(2m + w\)
dimensional normal distribution. The joint density of the model is then
represented by: \begin{equation}\label{YDThetarjoint} 
f(\boldsymbol{\theta}, \boldsymbol{r},
\boldsymbol{y}, \boldsymbol{d}) = 2^w\phi_{2m+w}((\boldsymbol{s}^t,
\boldsymbol{y}^t)^t \mid \boldsymbol{M} + (\boldsymbol{0}_{2m}^t, ({\rm
diag}(\boldsymbol{\lambda})\boldsymbol{d})^t)^t, \boldsymbol{\Sigma})
\phi_w(\boldsymbol{d} \mid \boldsymbol{0}_w, \boldsymbol{I}_w) \prod_{j =
1}^{m}r_j, 
\end{equation} \noindent where
\(\boldsymbol{s} = (r_1(\cos(\theta_1), \sin(\theta_1)), \dots, r_m(\cos(\theta_m), \sin(\theta_m)))^t\),
the mean vector
\(\boldsymbol{M} = (\boldsymbol{M}_s^t, \boldsymbol{M}_y^t)^t\) and
\(\boldsymbol{\Sigma} = \left ( \begin{matrix} \boldsymbol{\Sigma}_s & \boldsymbol{\Sigma}_{sy} \\ \boldsymbol{\Sigma}_{sy}^t & \boldsymbol{\Sigma}_y \\ \end{matrix} \right )\).
The matrix \(\boldsymbol{\Sigma}_s\) is the covariance matrix for the
variances of and covariances between the augmented representations of
the circular component and the matrix \(\boldsymbol{\Sigma}_{sy}\)
contains covariances between the augmented representations of the
circular component and the linear component. \newline \indent In our
regression extension we have \(i = 1, \dots, n\) observations of \(m\)
circular components, \(w\) linear components and \(g\) covariates. The
mean in the density in \eqref{YDThetarjoint} then becomes
\(\boldsymbol{M}_i = \boldsymbol{B}^t\boldsymbol{x}_i\) where
\(\boldsymbol{B}\) is a \((g + 1) \times (2m + w)\) matrix with
regression coefficients and intercepts and \(\boldsymbol{x}_i\) is a
\(g + 1\) dimensional vector containing the value 1 to estimate an
intercept and the \(g\) covariate values. \newline

\section{Model fit}

\indent We use the following (conditional) loglikelihoods for the
computation of the PLSL in the teacher data:

\begin{itemize}
\item For the modified CL-PN model:

$$l(y \mid \theta, r) = \log(1) - \log(\sqrt{2\pi\sigma^2}) + \sum(\hat{y}_i-(\gamma_0 + \gamma_{cos}\cos(\theta_i)r_i +
\gamma_{sin}\sin(\theta_i)r_i + \gamma_1\text{SE}_i))^2/2\sigma^2$$

$$l(\theta, r) = \log(1) - \log(2\pi) + \sum -0.5\hat{\boldsymbol{\mu}}_i^2 - 0.5(r_i^2 - 2r_iu_i^t\hat{\boldsymbol{\mu}}_i)$$
where $u_i = (cos\theta_i, \sin \theta_i)$ and $\hat{\boldsymbol{\mu}}_i = (\beta_0^{I} + \beta_0^{I}\text{SE}_i, \beta_0^{II} + \beta_0^{II}\text{SE}_i)^t$.

\item For the modified CL-GPN model:

$$l(y \mid \theta, r) = \log(1) - \log(\sqrt{2\pi\sigma^2}) + \sum(\hat{y}_i-(\gamma_0 + \gamma_{cos}\cos(\theta_i)r_i +
\gamma_{sin}\sin(\theta_i)r_i + \gamma_1\text{SE}_i))^2/2\sigma^2$$

$$l(\theta, r) =  \log(1) - \log(2\pi+\tau) - \sum \log(r_i) + (u_i^t\hat{\boldsymbol{\mu}}_i\Sigma^{-1}(u_i^t\hat{\boldsymbol{\mu}}_i)^t)/2\tau^2$$
where $u_i = (cos\theta_i, \sin \theta_i)$ and $\hat{\boldsymbol{\mu}}_i = (\beta_0^{I} + \beta_0^{I}\text{SE}_i, \beta_0^{II} + \beta_0^{II}\text{SE}_i)^t$.


\item For the modified Abe-Ley model:


$$l(y \mid \theta) = \log\alpha + \sum\log h_i^\alpha + \sum\log y_i^{\alpha - 1} - \sum(h_iy_i)^\alpha$$

where $h_i = \exp(\hat{y}_i)\{1-\tanh(\kappa)\cos(\theta_i - \hat{\theta}_i)\}^{1/\alpha}$, $\hat{y}_i = \gamma_0 + \gamma_1\text{SE}_i$ and $\hat{\theta}_i = \beta_0 +  2\tan^{-1}(\beta_1\text{SE}_i))$ .

$$l(\theta \mid y) = \log(1) - \sum\log 2\pi I_0(c_i) + \sum\log\{1 + \lambda \sin(\theta_i - \hat{\theta}_i\} + \sum c_i\cos(\theta_i - \hat{\theta}_i)$$

where $c_i = y_i^{\alpha}\exp(\hat{y}_i)^{\alpha}\tanh\kappa$, and $I_{0}$ is a modified Bessel function of order $0$.


\item For the modified joint projected and skew normal model we take the loglikelihoods of the following distributions:

$y_i \mid \boldsymbol{M}_i, \boldsymbol{\Sigma}, \theta_i, r_i \sim SSN(M_{i_y} + \lambda d_i + \boldsymbol{\Sigma}_{sy}^t\boldsymbol{\Sigma}_s^{-1}(\boldsymbol{s}_i - \boldsymbol{M}_{i_s}), \sigma^2_y + \boldsymbol{\Sigma}_{sy}^t\boldsymbol{\Sigma}_s^{-1}\boldsymbol{\Sigma}_{sy}),$

$\theta_i \mid \boldsymbol{M}_i, \boldsymbol{\Sigma}, y_i, d_i \sim GPN(\boldsymbol{M}_{i_s} + \boldsymbol{\Sigma}_{sy}\sigma^{-2}_y(y_i - M_{i_y} - \lambda d_i), \boldsymbol{\Sigma}_s + \boldsymbol{\Sigma}_{sy}\sigma_y^{-2}\boldsymbol{\Sigma}_{sy}^t)$

where $SSN$ is the skew normal distribution. Computationally this comes down to taking the log of the density values for a univariate and multivariate normal distribution (with mean and variance specified as above) for the linear and circular component respectively.

\end{itemize}

\section{MCMC procedures}
\subsection{Bayesian Model and MCMC procedure for the modified CL-PN model}\label{A1}

We use the following algorithm to obtain posterior estimates from the
model:

\begin{enumerate}
\item Split the data, with the circular component $\boldsymbol{\theta} = \theta_1, \dots, \theta_n$ and the linear component $\boldsymbol{y} = y_1, \dots, y_n$ where $n$ is the sample size, and the design matrices $\boldsymbol{Z}^k_{n \times 2}$ (for $k \in \{I,II\}$) and $\boldsymbol{X}_{n \times 4}$ of the circular and the linear component respectively, in a training (90\%) and holdout (10\%) set. 
\item Define the prior parameters for the training set. In this paper we use:

\begin{itemize}
\item Prior for $\boldsymbol{\gamma}$: $N_4(\boldsymbol{\mu}_{0}, \boldsymbol{\Lambda}_{0})$, with  $\boldsymbol{\mu}_{0} = (0,0,0,0)^t$ and  $\boldsymbol{\Lambda}_{0} = 10^{-4}\boldsymbol{I}_4$.
\item Prior for $\sigma^2$: $IG(\alpha_{0}, \beta_{0})$, an inverse gamma prior with $\alpha_{0} = 0.001$ and  $\beta_{0} = 0.001$.
\item Prior for $\boldsymbol{\beta^{k}}$: $N_2(\boldsymbol{\mu}_{0}, \boldsymbol{\Lambda}_{0})$, with $\boldsymbol{\mu}_{0} = (0,0)^t$ and  $\boldsymbol{\Lambda}_{0} = 10^{-4}\boldsymbol{I}_2$ for $k \in \{I,II\}$.
\end{itemize}

\item Set starting values $\boldsymbol{\gamma} = (0,0,0,0)^t$, $\sigma^2 = 1$ and $\boldsymbol{\beta^{k}} = (0,0)^t$ for $k \in \{I,II\}$. Also set starting values $r_i = 1$ in the training and holdout set. 
\item Compute the latent bivariate scores $\boldsymbol{s}_i = (s_i^{I}, s_i^{II})^t$ underlying the circular component for the holdout and training dataset as follows:
$$\begin{bmatrix} s^{I}_{i} \\ s^{II}_{i} \end{bmatrix} = \begin{bmatrix} r_i \cos (\theta_i)\\  r_i\sin (\theta_i)\end{bmatrix}.$$
\item Sample $\boldsymbol{\gamma}$, $\sigma^2$ and $\boldsymbol{\beta^{k}}$ for $k \in \{I,II\}$ for the training dataset from their conditional posteriors:

\begin{itemize}
\item Posterior for $\boldsymbol{\gamma}$: $N_4(\boldsymbol{\mu}_n, \sigma^2\boldsymbol{\Lambda}^{-1}_n)$, with $\boldsymbol{\mu}_n = (\boldsymbol{X}^t\boldsymbol{X} + \boldsymbol{\Lambda}_0)^{-1}(\boldsymbol{\Lambda}_0\boldsymbol{\mu}_0 + \boldsymbol{X}^t\boldsymbol{y})$ and $\boldsymbol{\Lambda}_n = (\boldsymbol{X}^t\boldsymbol{X} + \boldsymbol{\Lambda}_0)$.
\item Posterior for $\sigma^2$: $IG(\alpha_{n}, \beta_{n})$, an inverse gamma posterior with $\alpha_{n} = \alpha_0 + n/2$ and $\beta_{n} = \beta_0 + \frac{1}{2}(\boldsymbol{y}^t\boldsymbol{y} + \boldsymbol{\mu}_{0}^t\boldsymbol{\Lambda}_0\boldsymbol{\mu}_{0} + \boldsymbol{\mu}_{n}^t\boldsymbol{\Lambda}_n\boldsymbol{\mu}_{n})$.
\item Posterior for $\boldsymbol{\beta^{k}}$: $N_2(\boldsymbol{\mu}_n, \boldsymbol{\Lambda}_n)$, with $\boldsymbol{\mu}_n = ((\boldsymbol{Z}^k)^t\boldsymbol{Z}^k + \boldsymbol{\Lambda}_0)^{-1}(\boldsymbol{\Lambda}_0\boldsymbol{\mu}_0 + (\boldsymbol{Z}^k)^t\boldsymbol{s}^k)$ and $\boldsymbol{\Lambda}_n = ((\boldsymbol{Z}^k)^t\boldsymbol{Z}^k + \boldsymbol{\Lambda}_0)$.
\end{itemize}

\item Sample new $r_i$ for the training and holdout dataset from the following posterior:
$$f(r_i \mid \theta_i, \boldsymbol{\mu}_i) \propto r_i \exp{\left(-\frac{1}{2}(r_i)^2 + b_ir_i\right)}$$ 
where $b_i = \begin{bmatrix} \cos (\theta_i) \\ \sin (\theta_i)\end{bmatrix}^t\boldsymbol{\mu}_i$, $\boldsymbol{\mu}_i = \boldsymbol{B}^t\boldsymbol{z_i}$ and $\boldsymbol{B} = (\boldsymbol{\beta}^{I}, \boldsymbol{\beta}^{II})$. 
\noindent We can sample from this posterior using a slice sampling technique (Cremers et al., 2018): 

\begin{itemize}
\item In a slice sampler the joint density for an auxiliary variable $v_{i}$ with $r_{i}$ is
$$p(r_{i}, v_{i}\mid \theta_{i}, \boldsymbol{\mu}_{i}=\boldsymbol{B}^t\boldsymbol{z}_{i}) \propto r_{i} \textbf{I}\left(0 < v_i < \exp\left\{ -\frac{1}{2}(r_{i} - b_{i})^2\right\}\right)\textbf{I}(r_i > 0).$$
\noindent The full conditional for $v_{i}$, $p(v_{i} \mid r_{i},\boldsymbol{\mu}_{i}, \theta_{i})$, is
$$U\left(0, \exp\left\{-\frac{1}{2}(r_{i} -  b _{i})^2\right\}\right)$$
and the full conditional for $r_i$, $p(r_{i} \mid v_{i},\boldsymbol{\mu}_{i}, \theta_{i})$, is proportional to
$$r_{i} \textbf{I}\left(b_{i} + \max\left\{-b_{i}, -\sqrt{-2\ln v_{i}}\right\} < r_{i} < b_{i} + \sqrt{-2\ln v_{i}}\right).$$
\noindent We thus sample $v_{i}$ from the uniform distribution specified above. Independently we sample a value $m$ from $U(0,1)$. We obtain a new value for $r_{i}$ by computing $ r_{i} = \sqrt{(r_{i_{2}}^{2}-r_{i_{1}}^{2})m + r_{i_{1}}^{2}}$ where $r_{i_{1}}=b_{i} +\max\left\{-b_{i}, -\sqrt{-2\ln v_{i}}\right\}$ and $ r_{i_{2}}= b_{i} + \sqrt{-2\ln v_{i}}$.
\end{itemize}
\item Compute the PLSL for the circular and linear component on the holdout set using the estimates of $\boldsymbol{\gamma}$, $\sigma^2$ and $\boldsymbol{\beta^{k}}$ for $k \in \{I,II\}$ for the training dataset.
\item Repeat steps 4 to 7 until the sampled parameter estimates have converged. We assess convergence visually using traceplots.
\end{enumerate}

\newpage
\subsection{Bayesian Model and MCMC procedure for the modified CL-GPN model}\label{A2}

We use the following algorithm to obtain posterior estimates from the
model:

\begin{enumerate}
\item Split the data, with the circular component $\boldsymbol{\theta} = \theta_1, \dots, \theta_n$ and the linear component $\boldsymbol{y} = y_1, \dots, y_n$ where $n$ is the sample size, and the design matrices $\boldsymbol{Z}_{n \times 2}$  and $\boldsymbol{X}_{n \times 4}$ of the circular and the linear component respectively, in a training (90\%) and holdout (10\%) set. 
\item Define the prior parameters for the training set. In this paper we use:

\begin{itemize}
\item Prior for $\boldsymbol{\gamma}$: $N_4(\boldsymbol{\mu}_{0}, \boldsymbol{\Lambda}_{0})$, with $\boldsymbol{\mu}_{0} = (0,0,0,0)^t$ and $\boldsymbol{\Lambda}_{0} = 10^{-4}\boldsymbol{I}_4$.
\item Prior for $\sigma^2$: $IG(\alpha_{0}, \beta_{0})$, an inverse gamma prior with $\alpha_{0} = 0.001$ and $\beta_{0} = 0.001$.
\item Prior for $\boldsymbol{\beta}_{j}$: $N_2(\boldsymbol{\mu}_{0}, \boldsymbol{\Lambda}_0)$, with $\boldsymbol{\mu}_{0} = (0,0)^t$ and  $\boldsymbol{\Sigma}_{0} = 10^{5}\boldsymbol{I}_2$ for $j \in \{0, \dots, p\}$ where $p$ is the number of covariates, 1, in $\boldsymbol{Z}$.
\item Prior for $\xi$: $N(\mu_0, \sigma^2)$, with $\mu_0 = 0$ and $\sigma^2 = 10^{4}$.
\item Prior for $\tau$: $IG(\alpha_{0}, \beta_{0})$, an inverse gamma prior with $\alpha_{0} = 0.01$ and $\beta_{0} = 0.01$.
\end{itemize}

\item Set starting values $\boldsymbol{\gamma} = (0,0,0,0)^t$, $\sigma^2 = 1$, $\boldsymbol{\beta}_j = (0,0)^t$ for $j \in \{0,1\}$, $\xi = 0$, $\tau = 1$ and $\boldsymbol{\Sigma} = \begin{bmatrix} \tau^2 + \xi^2 & \xi\\ \xi & 1 \end{bmatrix}$. Also set starting values $r_i = 1$ in the training and holdout set. 
\item Compute the latent bivariate scores $\boldsymbol{s}_i = (s_i^{I}, s_i^{II})^t$ underlying the circular component for the holdout and training dataset as follows:
$$\begin{bmatrix} s^{I}_{i} \\ s^{II}_{i} \end{bmatrix} = \begin{bmatrix} r_i \cos (\theta_i) \\  r_i\sin (\theta_i)\end{bmatrix}.$$
\item Sample $\boldsymbol{\gamma}$, $\sigma^2$, $\boldsymbol{\beta}_j$ for $j \in \{0,1\}$, $\xi$ and $\tau$ for the training dataset from their conditional posteriors:

\begin{itemize}
\item Posterior for $\boldsymbol{\gamma}$: $N_4(\boldsymbol{\mu}_n, \sigma^2\boldsymbol{\Lambda}^{-1}_n)$, with $\boldsymbol{\mu}_n = (\boldsymbol{X}^t\boldsymbol{X} + \boldsymbol{\Lambda}_0)^{-1}(\boldsymbol{\Lambda}_0\boldsymbol{\mu}_0 + \boldsymbol{X}^t\boldsymbol{y})$ and $\boldsymbol{\Lambda}_n = (\boldsymbol{X}^t\boldsymbol{X} + \boldsymbol{\Lambda}_0)$.
\item Posterior for $\sigma^2$: $IG(\alpha_{n}, \beta_{n})$, an inverse gamma posterior where $\alpha_{n} = \alpha_0 + n/2$ and $\beta_{n} = \beta_0 + \frac{1}{2}(\boldsymbol{y}^t\boldsymbol{y} + \boldsymbol{\mu}_{0}^t\boldsymbol{\Lambda}_0\boldsymbol{\mu}_{0} + \boldsymbol{\mu}_{n}^t\boldsymbol{\Lambda}_n\boldsymbol{\mu}_{n})$.
\item Posterior for $\boldsymbol{\beta}_j$: $N_2(\boldsymbol{\mu}_{j_{n}}, \boldsymbol{\Sigma}_{j_{n}})$, with $\boldsymbol{\mu}_{j_{n}} = \boldsymbol{\Sigma}_{j_{n}}\boldsymbol{\Sigma}^{-1}\Bigg(-\sum_{i=1}^{n}z_{i,j-1}\sum_{l\neq j}z_{i,l-1}\boldsymbol{\beta}_l + \sum_{i=1}^{n}z_{i,j-1}r_i\begin{bmatrix} \cos (\theta_i) \\ \sin (\theta_i)\end{bmatrix}\Bigg)$ and  $\boldsymbol{\Sigma}_{j_{n}} = \Big(\sum_{i=1}^{n}z_{i,j-1}^2\boldsymbol{\Sigma}^{-1}+\boldsymbol{\Lambda}_0\Big)^{-1}$ for $j \in \{0, \dots, p\}$ where $p$ is the number of covariates, 1, in $\boldsymbol{Z}$.
\item Posterior for $\xi$: $N(\mu_n, \sigma^2_n)$, with $\mu_n = \frac{\tau^{-2} \sum_{i=1}^{n}(s^{I}_{i} - \mu_i^{I})(s^{II}_{i} - \mu_i^{II}) + \mu_0\sigma_0^{-2}}{\tau^{-2}\sum_{i=1}^{n}(s^{II}_{i} - \mu_i^{II})^2 + \sigma_0^{-2}}$ and $\sigma_n^2 = \frac{1}{\tau^{-2}\sum_{i=1}^{n}(s^{II}_{i} - \mu_i^{II})^2 + \sigma_0^{-2}}$ where $\mu_i^{I} = (\boldsymbol{\beta}^{I})^t\boldsymbol{z}_i$ and $\mu_i^{II} = (\boldsymbol{\beta}^{II})^t\boldsymbol{z}_i$. 
\item Posterior for $\tau$: $IG(\alpha_n, \beta_n)$, an inverse gamma posterior with $\alpha_n = \frac{n}{2} + \alpha_0$ and $\beta_n = \sum\limits_{i = 1}^{n}(s^{I}_{i} - \{\mu_i^{I} + \xi(s^{II}_{i} - \mu_i^{II})\})^2 + \beta_0$
\end{itemize}

\item Sample new $r_i$ for the training and holdout dataset from the following posterior:
$$f(r_i \mid \theta_i, \boldsymbol{\mu}_i) \propto r_i \exp{\left\{-\frac{1}{2}A_i\bigg(r_i-\frac{B_i}{A_i}\bigg)^2\right\}}$$ 
where $B_i = \begin{bmatrix} \cos (\theta_i) \\ \sin (\theta_i)\end{bmatrix}^t\boldsymbol{\Sigma}^{-1}\boldsymbol{\mu}_i$, $\boldsymbol{\mu}_i = \boldsymbol{B}^t\boldsymbol{z_i}$, $\boldsymbol{B} = (\boldsymbol{\beta}^{I}, \boldsymbol{\beta}^{II})$ and $A_i = \begin{bmatrix} \cos (\theta_i) \\ \sin (\theta_i)\end{bmatrix}^t\boldsymbol{\Sigma}^{-1}\begin{bmatrix} \cos (\theta_i) \\ \sin (\theta_i)\end{bmatrix}$.
\noindent We can sample from this posterior using a slice sampling technique (Hernandez-Stumpfhauser et al. 2018):

\begin{itemize}
\item In a slice sampler the joint density for an auxiliary variable $v_{i}$ with $r_{i}$ is
$$p(r_{i}, v_{i}\mid \theta_{i}, \boldsymbol{\mu}_{i}=\boldsymbol{B}^t\boldsymbol{z}_{i}) \propto r_{i} \textbf{I}\bigg(0 < v_i < \exp\left\{ -\frac{1}{2} A_i\left(r_{i} - \frac{B_i}{A_i}\right)^2\right\}\bigg)\textbf{I}(r_i > 0).$$
\item The full conditional for $v_{i}$, $p(v_{i} \mid r_{i},\boldsymbol{\mu}_{i}, \boldsymbol{\Sigma}, \theta_{i})$, is
$$U\Bigg(0, \exp\left\{-\frac{1}{2}A_i\bigg(r_i -  \frac{B_{i}}{A_i}\bigg)^2\right\}\Bigg)$$
and the full conditional for $r_i$, $p(r_{i} \mid v_{i},\boldsymbol{\mu}_{i}, \boldsymbol{\Sigma}, \theta_{i})$, is proportional to
$$r_{i} \textbf{I}\left(\frac{B_i}{A_i} + \max\left\{-\frac{B_i}{A_i}, -\sqrt{\frac{-2\ln v_{i}}{A_i}}\right\} < r_{i} < \frac{B_i}{A_i} + \sqrt{\frac{-2\ln v_{i}}{A_i}}\right).$$
\item We thus sample $v_{i}$ from the uniform distribution specified above. Independently we sample a value $m$ from $U(0,1)$. We obtain a new value for $r_{i}$ by computing $r_{i} = \sqrt{(r_{i_{2}}^{2}-r_{i_{1}}^{2})m + r_{i_{1}}^{2}}$ where $r_{i_{1}}=\frac{B_i}{A_i} +\max\left\{-\frac{B_i}{A_i}, -\sqrt{\frac{-2\ln v_{i}}{A_i}}\right\}$ and $ r_{i_{2}}= \frac{B_i}{A_i} + \sqrt{\frac{-2\ln v_{i}}{A_i}}$.
\end{itemize}

\item Compute the PLSL for the circular and linear component on the holdout set using the estimates of $\boldsymbol{\gamma}$, $\sigma^2$, $\boldsymbol{\beta^{k}}$ for $k \in \{I,II\}$, $\xi$ and $\tau$ for the training dataset. 
\item Repeat steps 4 to 7 until the sampled parameter estimates have converged. We visually assess convergence using traceplots.
\end{enumerate}

\newpage
\subsection{Bayesian Model and MCMC procedure for the modified GPN-SSN model}\label{A3}

\begin{enumerate}
\item Split the data, with the circular component $\boldsymbol{\theta} = \theta_1, \dots, \theta_n$ and the linear component $\boldsymbol{y} = y_1, \dots, y_n$ where $n$ is the sample size, and the design matrix $\boldsymbol{X}_{n \times 2}$ in a training (90\%) and holdout (10\%) set. Note that in this paper we have only one circular component and one linear component and the MCMC procedure outlined here is specified for this situation. It can however be generalized to a situation with multiple circular and linear components without too much effort. 
\item Define the prior parameters for the training set. Since we have only one circular component, one linear component and one covariate, we have $m = 1$, $w = 1$ and $g = 1$. In this paper we use the following priors:

\begin{itemize}
\item Prior for $\boldsymbol{\Sigma}$: $IW(\boldsymbol{\Psi}_0, \nu_0)$, an inverse Wishart with $\boldsymbol{\Psi}_0 = 10^{-4}\boldsymbol{I}_{2m + w}$ and $\nu_0 = 1$.   
\item Prior for $\boldsymbol{B}$ in vectorized form: $N_{(g + 1)(2m + w)}(\boldsymbol{\beta}_0, \boldsymbol{\Sigma}  \otimes \boldsymbol{\kappa}_0)$, where $\otimes$ stands for the Kronecker product, $\boldsymbol{\beta}_0 = \text{vec}(\boldsymbol{B}_0)$, the matrix with prior values for the regression coefficients. We choose $\boldsymbol{\beta}_0 = \boldsymbol{0}_{(g + 1)(2m + w)}$, $\boldsymbol{B}_0 = \boldsymbol{0}_{(g + 1) \times (2m + w)}$ and $\boldsymbol{\kappa}_0 = 10^{-4}\boldsymbol{I}_{g + 1}$.
\item Prior for $\lambda$: $N(\gamma_0, \omega_0)$, with $\gamma_0 = 0$ and $\omega_0 = 10000$.
\end{itemize}

\item Set starting values $\boldsymbol{\beta} = (0,0,0,0,0,0)^t$, $\boldsymbol{\Sigma} = \boldsymbol{I}_3$ and $\lambda = 0$. Also set starting values $r_i = 1$ and $d_i = 1$ in the training and holdout set. 
\item Compute the latent bivariate scores $\boldsymbol{s}_i = (s_i^{I}, s_i^{II})^t$ underlying the circular component for the holdout and training dataset as follows:
$$\begin{bmatrix} s^{I}_{i} \\ s^{II}_{i} \end{bmatrix} = \begin{bmatrix} r_i \cos (\theta_i) \\  r_i\sin (\theta_i)\end{bmatrix}.$$
\item Compute the latent scores $\tilde{y}_i$ underlying the linear component for the holdout and training dataset as follows:
$$\tilde{y}_i = \lambda d_i. $$
\item Compute $\boldsymbol{\eta}_i$ defined as follows for each individual $i$:
$$\boldsymbol{\eta}_i = (\boldsymbol{s}_i^t,y_i)^t - (\boldsymbol{0}_{2m}^t, \lambda d_i)^{t}.$$

\item Sample $\boldsymbol{B}$, $\boldsymbol{\Sigma}$ and $\lambda$ for the training dataset from their conditional posteriors: 

\begin{itemize}
\item Posterior for $\boldsymbol{\Sigma}$: $IW(\boldsymbol{\Psi}_n, \nu_n)$, an inverse Wishart with $\boldsymbol{\Psi}_n = \boldsymbol{\Psi}_0 + (\boldsymbol{\eta} - \boldsymbol{X}^t\boldsymbol{B})^t(\boldsymbol{\eta} - \boldsymbol{X}^t\boldsymbol{B}) + (\boldsymbol{B} - \boldsymbol{B}_0)^t\boldsymbol{\kappa}_0(\boldsymbol{B} - \boldsymbol{B}_0)$ and $\nu_n = \nu_0 + n$ where $n$ is the sample size.
\item Posterior for $\boldsymbol{B}$ in matrix form: $MN(\boldsymbol{B}_n,   \boldsymbol{\kappa}_n, \boldsymbol{\Sigma})$, with $\boldsymbol{B}_n = \boldsymbol{\kappa}_n^{-1}\boldsymbol{X}^t\boldsymbol{\eta} + \boldsymbol{\kappa}_0\boldsymbol{B}_0$ and $\boldsymbol{\kappa}_n = \boldsymbol{X}^t\boldsymbol{X} + \boldsymbol{\kappa}_0$.
\item Posterior for $\lambda$: $N(\gamma_n, \omega_n)$, with $\omega_n = \big(\sum_{i = 1}^{n}d_i^2\sigma^{-2}_{y|s} + \omega_0^{-1}\big)^{-1}$ and $\gamma_n = \omega_n \big(\sum_{i = 1}^{n}d_i\sigma^{-2}_{y|s}(y_i - \mu_{y_i|s_i}) + \omega_0^{-1}\gamma_0 \big)$ where $\mu_{y_i|s_i} = \mu_y + \boldsymbol{\Sigma}_{sy}^{t}\boldsymbol{\Sigma}_{s}^{-1}(\boldsymbol{s_i - \boldsymbol{\mu}_s)}$ and $\sigma^2_{y|s} = \sigma^2_{y} - \boldsymbol{\Sigma}_{sy}^{t}\boldsymbol{\Sigma}_{s}^{-1}\boldsymbol{\Sigma}_{sy}$.
\end{itemize}

\item Sample new $d_i$ for the training and holdout dataset from the following posterior:
$$f(d_i) \propto \phi(y_i|\mu_{y_i|s_i} + \lambda d_i, \sigma^2_{y|s})\phi(d_i|0, 1),$$
where $\mu_{y_i|s_i} = \boldsymbol{B}_{y_i|s_i}^t\boldsymbol{x}_i$. We can see each $d_i$ as a positive regressor with $\lambda$ as covariate and $\phi(d_i|0, 1)$ as prior (Mastrantonio, 2018). The full conditional is then truncated normal with support $\mathbb{R}^{+}$ as follows:
$$N(m_{d_i}, v),$$ 
\noindent where $v = \big(\lambda^2\sigma^{-2}_{y|s} + 1\big)$ and $m_{d_i} = v\lambda\sigma^{-2}_{y|s}\big(y_i - \mu_{y_i|s_i}\big)$. 
\item Sample new $r_i$ for the training and holdout dataset from the following posterior
$$f(r_i \mid \theta_i, \boldsymbol{\mu}_i) \propto r_i \exp{\left\{-0.5A_i\bigg(r_i-\frac{B_i}{A_i}\bigg)^2\right\}}$$ 
where $B_i = \begin{bmatrix} \cos (\theta_i) \\ \sin (\theta_i)\end{bmatrix}^t\boldsymbol{\Sigma}_{s_i|y_i}^{-1}\boldsymbol{\mu}_{s_i|y_i}$, $\boldsymbol{\mu}_{s_i|y_i} = \boldsymbol{B}_{s_i|y_i}^t\boldsymbol{x}_i$ and $A_i = \begin{bmatrix} \cos (\theta_i) \\ \sin (\theta_i)\end{bmatrix}^t\boldsymbol{\Sigma}_{s_i| y_i}^{-1}\begin{bmatrix} \cos (\theta_i) \\ \sin (\theta_i)\end{bmatrix}$. The parameters $\boldsymbol{\mu}_{s_i|y_i}$ and $\boldsymbol{\Sigma}_{s_i| y_i}$ are the conditional mean and covariance matrix of $\boldsymbol{s}_i$ assuming that $(\boldsymbol{s}_i^t, y_i)^t \sim N_{2m+w}(\boldsymbol{\mu} + (\boldsymbol{0}_{2m}^t, \lambda d_i)^t, \boldsymbol{\Sigma})$. 
Because in this paper $\boldsymbol{\theta}$ originates from a bivariate variable that is known we can in this model (where the variance-covariance matrix of the circular component is not constrained in the estimation procedure) simply define the $r_i$ as the Euclidean norm of the bivariate datapoints. However, for didactic purposes we continue with the explanation of the sampling procedure. We
can sample from the posterior for $r_i$ using a slice sampling technique (Hernandez-Stumpfhauser et al. 2018): 
\begin{itemize}
\item In a slice sampler the joint density for an auxiliary variable $v_{i}$ with $r_{i}$ is
$$p(r_{i}, v_{i}\mid \theta_{i}, \boldsymbol{\mu}_{i}=\boldsymbol{B}^{t}\boldsymbol{x}_{i}) \propto r_{i} \textbf{I}\bigg(0 < v_i < \exp\left\{ -\frac{1}{2}A_i\left(r_{i} - \frac{B_i}{A_i}\right)^2\right\}\bigg)\textbf{I}(r_i > 0).$$
\item The full conditional for $v_{i}$, $p(v_{i} \mid r_{i},\boldsymbol{\mu}_{i}, \boldsymbol{\Sigma}, \theta_{i})$, is
$$U\Bigg(0, \exp\left\{-\frac{1}{2}A_i\bigg(r_i -  \frac{B_{i}}{A_i}\bigg)^2\right\}\Bigg)$$
and the full conditional for $r_i$, $p(r_{i} \mid v_{i},\boldsymbol{\mu}_{i}, \boldsymbol{\Sigma}, \theta_{i})$, is proportional to 
$$r_{i} \textbf{I}\left(\frac{B_i}{A_i} + \max\left\{-\frac{B_i}{A_i}, -\sqrt{\frac{-2\ln v_{i}}{A_i}}\right\} < r_{i} < \frac{B_i}{A_i} + \sqrt{\frac{-2\ln v_{i}}{A_i}}\right)$$
\item We thus sample $v_{i}$ from the uniform distribution specified above. Independently we sample a value $m$ from $U(0,1)$. We obtain a new value for $r_{i}$ by computing $r_{i} = \sqrt{(r_{i_{2}}^{2}-r_{i_{1}}^{2})m + r_{i_{1}}^{2}}$ where $r_{i_{1}}=\frac{B_i}{A_i} +\max\left\{-\frac{B_i}{A_i}, -\sqrt{\frac{-2\ln v_{i}}{A_i}}\right\}$ and $ r_{i_{2}}= \frac{B_i}{A_i} + \sqrt{\frac{-2\ln v_{i}}{A_i}}$.

\end{itemize}

\item Compute the PLSL for the circular and linear component on the holdout set using the estimates of $\boldsymbol{B}$, $\boldsymbol{\Sigma}$ and $\lambda$ for the training dataset.

\item Repeat steps 4 to 10 until the sampled parameter estimates have converged.

\item In the MCMC sampler we have estimated an unconstrained $\boldsymbol{\Sigma}$. However, for identification of the model we need to apply  constraints to both $\boldsymbol{\Sigma}$ and $\boldsymbol{\mu}$. Therefore we need the matrix
$$\boldsymbol{C} = \begin{bmatrix} \boldsymbol{C}_s & \boldsymbol{0}_{2m \times w} \\ \boldsymbol{0}_{2m \times w}^t & \boldsymbol{I}_w \end{bmatrix}$$
where $\boldsymbol{C}_s$ is a $2m \times 2m$ diagonal matrix with every $(2(j-1) + k)^{th}$ entry $> 0$ where $k \in \{1,2\}$ and $j = 1, \dots, m$ (Mastrantonio, 2018). The estimates $\boldsymbol{\Sigma}$ and $\boldsymbol{\mu}$ can then be related to their constrained versions $\tilde{\boldsymbol{\Sigma}}$ and $\tilde{\boldsymbol{\mu}}$ as follows:
$$\boldsymbol{\mu} = \boldsymbol{C}\tilde{\boldsymbol{\mu}}$$
$$\boldsymbol{\Sigma} = \boldsymbol{C}\tilde{\boldsymbol{\Sigma}}\boldsymbol{C}.$$

\end{enumerate}
\newpage
\section*{References}

\hypertarget{refs}{}
\leavevmode\hypertarget{ref-abe2017tractable}{}%
Abe, T., \& Ley, C. (2017). A tractable, parsimonious and flexible model
for cylindrical data, with applications. \emph{Econometrics and
Statistics}, \emph{4}, 91--104.
doi:\href{https://doi.org/10.1016/j.ecosta.2016.04.001}{10.1016/j.ecosta.2016.04.001}

\leavevmode\hypertarget{ref-hernandez2016general}{}%
Hernandez-Stumpfhauser, D., Breidt, F. J., \&
\VANDER{Woerd}{Van der}{van der} Woerd, M. J. (2016). The general
projected normal distribution of arbitrary dimension: Modeling and
Bayesian inference. \emph{Bayesian Analysis}, \emph{12}(1), 113--133.
doi:\href{https://doi.org/10.1214/15-BA989}{10.1214/15-BA989}

\leavevmode\hypertarget{ref-mastrantonio2018joint}{}%
Mastrantonio, G. (2018). The joint projected normal and skew-normal: A
distribution for poly-cylindrical data. \emph{Journal of Multivariate
Analysis}, \emph{165}, 14--26.
doi:\href{https://doi.org/10.1016/j.jmva.2017.11.006}{10.1016/j.jmva.2017.11.006}

\leavevmode\hypertarget{ref-mastrantonio2015bayesian}{}%
Mastrantonio, G., Maruotti, A., \& Jona-Lasinio, G. (2015). Bayesian
hidden Markov modelling using circular-linear general projected normal
distribution. \emph{Environmetrics}, \emph{26}(2), 145--158.
doi:\href{https://doi.org/10.1002/env.2326}{10.1002/env.2326}

\leavevmode\hypertarget{ref-nunez2011bayesian}{}%
Nuñez-Antonio, G., Gutiérrez-Peña, E., \& Escarela, G. (2011). A
Bayesian regression model for circular data based on the projected
normal distribution. \emph{Statistical Modelling}, \emph{11}(3),
185--201.
doi:\href{https://doi.org/10.1177/1471082X1001100301}{10.1177/1471082X1001100301}

\leavevmode\hypertarget{ref-olmos2012extension}{}%
Olmos, N. M., Varela, H., Gómez, H. W., \& Bolfarine, H. (2012). An
extension of the half-normal distribution. \emph{Statistical Papers},
\emph{53}(4), 875--886.
doi:\href{https://doi.org/10.1007/s00362-011-0391-4}{10.1007/s00362-011-0391-4}

\leavevmode\hypertarget{ref-sahu2003new}{}%
Sahu, S. K., Dey, D. K., \& Branco, M. D. (2003). A new class of
multivariate skew distributions with applications to Bayesian regression
models. \emph{Canadian Journal of Statistics}, \emph{31}(2), 129--150.
doi:\href{https://doi.org/10.2307/3316064}{10.2307/3316064}

\leavevmode\hypertarget{ref-wang2012directional}{}%
Wang, F., \& Gelfand, A. E. (2013). Directional data analysis under the
general projected normal distribution. \emph{Statistical Methodology},
\emph{10}(1), 113--127.
doi:\href{https://doi.org/10.1016/j.stamet.2012.07.005}{10.1016/j.stamet.2012.07.005}


\end{document}
